\documentclass{article}
\usepackage[utf8]{inputenc}
\usepackage{amsfonts}
\usepackage{amsmath}
\usepackage{graphicx}
\usepackage[a4paper, total={6in, 8in}]{geometry}
\usepackage{setspace}

\newcommand\tab[1][1cm]{\hspace*{#1}}
\onehalfspacing
\author{Frederic Becerril}

\begin{document}

$$ \| P - f \| \leq \frac{ \| f^{ (3) } \| }{ 3! } \| \pi_3 \|. $$ On peut majorer $ \| \pi_3 \| \leq 1 $. Comme $ \left( e^{ -1/3 x } \right)''' = - \frac{ 1 }{ 27 } e^{ -1/3 x } $, on a $ \| P - f \| \leq \frac{ 1 }{ 27 } \cdot \frac{ 1 }{ 6 } = \frac{ 1 }{ 162 } $. Comme $ \frac{ 1 }{ 162 } < \frac{ 1 }{ 100 } $, cela veut dire que $ P $ peut servir pour calculer la valeur de $ f $ sur l'intervalle $ [ 0, 1 ] $ avec la précision de $ 2 $ chiffres après la virgule.


\end{document}
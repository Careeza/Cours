\documentclass{article}
\usepackage[utf8]{inputenc}
\usepackage{amsfonts}
\usepackage{amsmath}
\usepackage{graphicx}
\usepackage[a4paper, total={6in, 8in}]{geometry}
\usepackage{setspace}

\newcommand\tab[1][1cm]{\hspace*{#1}}
\onehalfspacing
\author{Frederic Becerril}

\begin{document}

\part*{Exerice 2}

\begin{itemize}
    \item Symétrie: Soit $X, Y \in \mathbb{R}^n$\\
    \begin{align*}
        <X|Y> = \sum_{k=1}^n x_k y_k = \sum_{k=1}^n y_k x_k = <Y | X>
    \end{align*}
    \item Bilinéarité: Soit $\alpha, \beta \in \mathbb{R} \mbox{ et } X, Y, Z \in \mathbb{R}^n$
    \begin{center}
        \begin{minipage}{0.50\textwidth}    
            \begin{itemize}
                \item[$ <\alpha X + \beta Y | Z>$] $= \sum_{k=1}^n (\alpha x_k + \beta y_k) * z_k$
                \item[] $= \sum_{k=1}^n \alpha x_k z_k + \beta y_k z_k$
                \item[] $= \sum_{k=1}^n \alpha x_k z_k + \sum_{k=1}^n \beta y_k z_k$
                \item[] $= \alpha \sum_{k=1}^n x_k z_k + \beta \sum_{k=1}^n y_k z_k$
                \item[] $= \alpha <X | Z> + \beta <Y | Z> $
            \end{itemize}
        \end{minipage}
    \end{center}
    Comme on a prouvé la symétrie on a que
    \begin{align*}
        <X | \alpha Y + \beta Z> = <\alpha Y + \beta Z | X>
    \end{align*}
    Ce qui termine la preuve de la bilinéarité
    \item Défini positif: Soit $X \in \mathbb{R}^n$
    \begin{center}
        \begin{minipage}{0.50\textwidth}    
            \begin{itemize}
                \item[$<X | X>$] $=\sum_{k=1}^n x_k x_k$
                \item[] $= \sum_{k=1}^n (x_k)^2$ Or comme $x_k \in \mathbb{R}, (x_k)^2 \geq 0$
                \item[] $\geq 0$
            \end{itemize}
        \end{minipage}
    \end{center}
    Si on a $<X | X> = 0$ alors:
    \begin{center}
        \begin{minipage}{0.50\textwidth}    
            \begin{itemize}
                \item[$<X | X> = 0$] $\Leftrightarrow \sum_{k=1}^n (x_k)^2 = 0$
                \item[] $\Leftrightarrow (x_k)^2 = 0 \; \forall 1 \leq k \leq n$
                \item[] $\Leftrightarrow x_k = 0 \; \forall 1 \leq k \leq n$
                \item[] $\Leftrightarrow X = 0$   
            \end{itemize}
        \end{minipage}
    \end{center}
\end{itemize}

\newpage


\begin{itemize}
    \item[$||X + Y||^2$] $=<X + Y | X + Y>$
    \item[] $=<X | X> + <Y | Y> + 2 <X | Y>$
    \item[] $= ||X||^2 + ||Y||^2 + 2 <X | Y>$
    \item[] $\leq ||X||^2 + ||Y||^2 + 2 |<X | Y>|$
    \item[] $\leq  ||X||^2 + ||Y||^2 + 2 (||X|| . ||Y||)$ Inégalité de Cauchy-Schwarz
    \item[] $=(||X|| + ||Y||)^2$ 
\end{itemize}
Or comme $||X + Y||^2 \geq 0$ et $(||X|| + ||Y||)^2 \geq 0$\\
On a que $||X + Y|| \leq ||X|| + ||Y||$\\
Montrons que $||x||_2$ est une norme:\\
$\bullet \; ||x||_2 = \sqrt{<x|x>}$ or $<x|x>$ est défini positif donc on a bien:
\begin{align*}
    \mathbb{R}^n \rightarrow \mathbb{R}_+\\
    X \longmapsto ||X||_2
\end{align*}
$\bullet$ Soit $\alpha \in \mathbb{R}$ et $X \in \mathbb{R}^n$
\begin{center}
    \begin{minipage}{0.50\textwidth}    
        \begin{itemize}
            \item[$||\alpha X||$] $= \sqrt{<\alpha X | \alpha X>}$
            \item[] $= \sqrt{\alpha^2 <X | X>}$ (*)
            \item[] $= |\alpha| \sqrt{<x|x>}$
            \item[] $= |\alpha| ||x||_2$   
        \end{itemize}
    \end{minipage}
\end{center}
(*) Grâce à la bilinéarité du produit sclaire\\
$\bullet$ On a montré à la question précendente l'inégalité triangulaire
$\bullet$ $Soit X \in \mathbb{R}^n$
\begin{center}
    \begin{minipage}{0.50\textwidth}    
        \begin{itemize}
            \item[$||X||_2 = 0$] $\Leftrightarrow \sqrt{<X|X>} = 0$
            \item[] $\Leftrightarrow <X|X> = 0$
            \item[] $\Leftrightarrow X = 0_E$ (**)          
        \end{itemize}
    \end{minipage}
\end{center}
(**) Car le produit sclaire est défini positif

\end{document}
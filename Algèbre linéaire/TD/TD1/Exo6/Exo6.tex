\documentclass{article}
\usepackage[utf8]{inputenc}
\usepackage{amsfonts}
\usepackage{amsmath}
\usepackage{graphicx}
\usepackage[a4paper, total={6in, 8in}]{geometry}
\usepackage{setspace}

\newcommand\tab[1][1cm]{\hspace*{#1}}
\onehalfspacing
\author{Frederic Becerril}

\begin{document}

\part*{Exerice 6}

E = $C^0([-1, 1], \mathbb{R})$\\
$\varphi: E^2 \rightarrow \mathbb{R}$\\
$(f, g) \longmapsto \int_{-1}^1 f(t)g(t)dt$
\begin{itemize}
    \item[\underline{Symétrie}] $Soit f, g \in E$\\
$\varphi(f, g) = \int_{-1}^1 f(t)g(t)dt = \int_{-1}^1 g(t)f(t)dt = \varphi(g, f) \tab \checkmark$
    \item[\underline{Bilinéarité}] $Soit f, g, h \in E \mbox{ et } \alpha, \beta \in \mathbb{R}$\\
    $\varphi(\alpha f + \beta g, h) = \int_{-1}^1 (\alpha f + \beta g)(t)h(t)dt$\\
    $\tab[2.2cm]= \int_{-1}^1 (\alpha f)(t)h(t) + (\beta g)(t)h(t)dt$\\
    $\tab[2.2cm]= \int_{-1}^1 \alpha f(t)h(t) + \beta g(t)h(t)dt$\\
    $\tab[2.2cm]= \int_{-1}^1 \alpha f(t)h(t) + \int_{-1}^1 \beta g(t)h(t)dt$\\
    $\tab[2.2cm]= \alpha \int_{-1}^1 f(t)h(t) + \beta \int_{-1}^1 g(t)h(t)dt$\\
    $\tab[2.2cm]= \alpha \varphi(f, h) + \beta \varphi(g, h)$\\
    Commme on a montré la symétrie cela suffit pour montrer la bilinéarité $\checkmark$
    \item[\underline{Définis}] $Soit f \in E$\\
    $\varphi(f, f) = \int_{-1}^1 f(t)f(t) = \int_{-1}^1 f(t)^2$ or $f(t)^2 \geq 0$ donc $ \int_{-1}^1 f(t)^2 \geq 0$\\
    Si $\varphi(f, f) = 0 \Leftrightarrow  \int_{-1}^1 f(t)^2 = 0$ or comme $\forall t \in [-1, 1] f(t)^2 \geq 0$\\
    si on veut que l'intégral soit nul Il faut que $f(t)^2 = 0$ $\forall t \in [-1, 1]$\\
    donc $f(t) = 0, \; \forall t \in [-1, 1] \Leftrightarrow f(t) = 0_E \tab \checkmark$
    
\end{itemize}
\end{document}
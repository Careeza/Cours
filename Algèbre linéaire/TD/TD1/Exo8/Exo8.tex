\documentclass{article}
\usepackage[utf8]{inputenc}
\usepackage{amsfonts}
\usepackage{amsmath}
\usepackage{graphicx}
\usepackage[a4paper, total={6in, 8in}]{geometry}
\usepackage{setspace}

\newcommand\tab[1][1cm]{\hspace*{#1}}
\onehalfspacing
\author{Frederic Becerril}

\begin{document}

\part*{Exerice 8}

On se place dans $\mathbb{R}^n$ muni du produit sclaire $<.|.>$ usuel\\
Soit $u, v \in \mathbb{R}^n$
\begin{itemize}
    \item[i)] $c \in \mathbb{R} \tab <u|cv> = c<u|v>$\\
    Vrai car le produit sclaire est bilinéaire
    \item[ii)] $c \in \mathbb{R} \tab ||cu|| = c||u||$\\
    Faux contre exemple $c = -1$ $||-u|| = -||u|| < 0$ impossible car une norme est toujours positif\\ 
    On a par contre $||cu|| = |c| * ||u||$
    \item[iii)] $<u|v> - <v|u> = 0$\\
    Vrai car le produit sclaire est symétrique donc $<u|v> = <v|u> \Leftrightarrow <u|v> - <v|u> = 0$
    \item[iv)] $||u + v||^2 = ||u||^2 + ||v||^2$ alors u et v sont orthogonaux\\
    Vrai c'est Pythagore\\
    Preuve: $||u + v||^2 = <u + v | u + v>$\\ 
    $\tab[2.8cm] = <u|u> + <u|v> + <v|u> + <v|v>$ Bilinéarité du produit sclaire\\
    $\tab[2.8cm] = ||u||^2 +||v||^2 + 2<u|v>$ or $<u|v> = 0$ ssi u et v sont orthogonaux\\
    $\tab[2.8cm]$ donc $||u + v||^2 = ||u||^2 + ||v||^2$ ssi u et v sont orthogonaux
    \item[v)] Un famille libre de $\mathbb{R}^n$ n'est pas forcement orthogonale\\
    Vrai exemple la famille $\{\begin{pmatrix}
        1\\
        1\\
    \end{pmatrix}, \begin{pmatrix}
        1\\
        0\\
    \end{pmatrix}\}$ de $\mathbb{R}^2$ est libre et $\left<
        \begin{pmatrix}
            1\\
            1\\
        \end{pmatrix} | \begin{pmatrix}
            1\\
            0\\
        \end{pmatrix}
    \right> = 1 * 1 + 1 * 0 = 1 \neq 0$
    \item[vi)] Une famille orthogonale de vecteurs de $\mathbb{R}^n$ n'est pas forcement libre.\\
    Vrai Si dans la famille orthogonal il y a des vecteurs nuls alors la famille n'est pas libre\\
    Par contre si la famille orthogonal n'a aucun vecteur nul, alors la famille est libre
    \item[vii)] Si on norme des vecteurs orthogonaux (non nuls), alors les vecteurs obtenus ne sont pas forcement orthogonaux.\\
    Faux Soit $u, v$ deux vecteurs orthogonaux non nul\\
    On a $<\frac{u}{||u||}|\frac{v}{||v||}> = \frac{<u|v>}{||u|| * ||v||} = \frac{0}{||u|| * ||v||} = 0$\\
    Donc normaliser ne change pas que deux vecteurs soit orthogonaux
    \item[viii)] Si W est un sous-espace vectoriel de $\mathbb{R}^n$ engendre par n vecteurs non nuls deux a deux orthogonaux, alors W = $\mathbb{R}^n$.\\
    Vrai on a vu au vi) que une famille orthogonale est libre donc W est famille de n vecteurs libre, donc W est une base de $\mathbb{R}^n$
\end{itemize}

\end{document}
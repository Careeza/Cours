\documentclass{article}
\usepackage[utf8]{inputenc}
\usepackage{amsfonts}
\usepackage{amsmath}
\usepackage{graphicx}
\usepackage[a4paper, total={6in, 8in}]{geometry}
\usepackage{setspace}

\newcommand\tab[1][1cm]{\hspace*{#1}}
\onehalfspacing
\author{Frederic Becerril}

\begin{document}

\part*{Exerice 5}

Soit $(E, <.|.>)$ un espace euclidien de dimension $n \geq 2$ et u un vecteur non nul de E\\
$V = \{x \in E, <x|u> = 0\}$\\
On veut montrer que V et $Vect\{u\}$ sont en somme directe\\
Soit $U = Vect\{u\}$
\begin{itemize}
    \item $U \cap V = 0_E$ ?\\
Soit $x \in U \cap V$ alors $x \in U$ et $x \in V$\\
Comme $x \in U, \exists \alpha \in \mathbb{R} \mbox{ tq } x = \alpha u$\\
Comme $x \in U$, on a $<x|u> = 0$\\
Donc on a $<\alpha u | u> = 0 \Leftrightarrow \alpha <u | u> = 0$ or u est un vecteur non nul donc $<u|u> \neq 0$\\
Finalement on a que $\alpha = 0$, donc que $x = 0$ et donc que $U \cap V = 0_E$\\
Donc U et V sont en somme directe\\
    \item Soit $\varphi$ une application de $E \rightarrow \mathbb{R}$, qui à tout vecteur x associe $<x|u>$\\
Soit $(x, y) \in E^2$ et $\alpha \in \mathbb{R}$\\
$\varphi(\alpha x + y) = <\alpha x + y | u> = \alpha <x | u> + <y | u> = \alpha \varphi(x) + \varphi(y)$\\
Donc $\varphi$ est une forme linéaire\\
$ker \varphi = \{x \in E, \varphi(x) = 0\} = \{x \in E, <x|u> = 0\} = V$\\
Comme $\varphi$ est une application linéaire on a que $dim(E) = dim(ker\varphi) + dim(im\varphi)$\\
Or $\varphi$ va de $E \rightarrow \mathbb{R}$, donc $dim(im \varphi) = 1$\\
Donc $dim(ker \varphi) = dim(V) = dim(E) - dim(im \varphi) = n - 1$
    \item On a vu que et U et V était en somme directe de plus $dim(U) + dim(V) = 1 + N - 1 = N = Dim(E)$, donc on a que $U \bigoplus V = E$\\
Donc on peut définir la projection orthogonal\\

\end{itemize}



\end{document}
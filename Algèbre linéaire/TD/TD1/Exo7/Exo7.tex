\documentclass{article}
\usepackage[utf8]{inputenc}
\usepackage{amsfonts}
\usepackage{amsmath}
\usepackage{graphicx}
\usepackage[a4paper, total={6in, 8in}]{geometry}
\usepackage{setspace}

\newcommand\tab[1][1cm]{\hspace*{#1}}
\onehalfspacing
\author{Frederic Becerril}

\begin{document}

\part*{Exerice 7}

$\bullet$ $\mathbb{R}_2[x]$ est un espace vectorielle de dimension 3 car une base $\mathbb{R}_2[x]$ est $Vect\{1, X, X^2\}$\\
$\bullet$ Soit $aX^2 + bX + c \in \mathbb{R}_2[x]$\\
On veut $\left\{
    \begin{array}{ll}
        P(0) = 0\\
        P(1) = 0\\
        P(2) = 0\\
    \end{array}
\right. \Leftrightarrow \left\{
    \begin{array}{ll}
        a * 0^2 + b * 0 + c = 0\\
        a * 1^2 + b * 1 + c = 0\\
        a * 2^2 + b * 2 + c = 0\\
    \end{array}
\right. \Leftrightarrow \left\{
    \begin{array}{ll}
        c = 0\\
        a + b = 0\\
        4a + 2b = 0\\
    \end{array}
\right. \Leftrightarrow \left\{
    \begin{array}{ll}
        c = 0\\
        a = -b\\
        -4b + 2b = 0\\
    \end{array}
\right.\\ \Leftrightarrow \left\{
    \begin{array}{ll}
        c = 0\\
        a = 0\\
        b = 0\\
    \end{array}
\right.$\\
$\bullet$ $f: \mathbb{R}_2[X]^2 \rightarrow \mathbb{R}$\\
$\tab[0.8cm] P, Q \longmapsto P(0)Q(0) + P(1)Q(1) + P(2)Q(2)$\\
\\
Montrons que f est un produit sclaire:\\
Symétire: soit $P, Q \in \mathbb{R}_2[X]$\\
$f(P, Q) =  P(0)Q(0) + P(1)Q(1) + P(2)Q(2) =  Q(0)P(0) + Q(1)P(1) + Q(2)P(2) = f(Q, P)$\\
\\
Bilinéarité: soit $P, Q, R \in \mathbb{R}_2[X]$ et $\alpha, \beta \in \mathbb{R}$\\
Comme nous avons prouvé la symétire il suffit de montre que:\\
$f(\alpha P + \beta Q, R) = f(R, \alpha P + \beta Q) = \alpha f(P, R) + \beta f(Q, R)$ pour la bilinéarité\\
$f(\alpha P + \beta Q, R) = (\alpha P + \beta Q)(0)R(0) + (\alpha P + \beta Q)(1)R(1) + (\alpha P + \beta Q)(2)R(2)$\\
$\tab[2.45cm] = \alpha P(0)R(0) + \beta Q(0)R(0) + \alpha P(1)R(1) + \beta Q(1)R(1) + \alpha P(2)R(2) + \beta Q(2)R(2)$
$\tab[2.45cm] = \alpha (P(0)R(0) + P(1)R(1) + P(2)R(2)) + \beta (Q(0)R(0) + Q(1)R(1) + Q(2)R(2)$\\
$\tab[2.45cm] = \alpha f(P, R) + \beta f(Q, R)$\\
\\
Définis positif: soit $P \in \mathbb{R}_2[X]$\\
$f(P, P) = P(0)P(0) + P(1)P(1) + P(2)P(2)$\\
$\tab[1.23cm] = P(0)^2 + P(1)^2 + P(2)^2 \geq 0$\\
$f(P, P) = 0 \Leftrightarrow P(0)^2 + P(1)^2 + P(2)^2 = 0$\\
$\tab[1.9cm] \Leftrightarrow P(0)^2 = P(1)^2 + P(2)^2 = 0$\\
$\tab[1.9cm] \Leftrightarrow P(0) = P(1) + P(2) = 0$\\
Or on a vu a la question 1 que le seul polynome respectant cette condition est le polynome nul\\
\\
$\bullet$ $f(1, X) = 1 * 0 + 1 * 1 + 1 * 1 = 2 \neq 0$\\
donc la famille $\{1, X, x^2\}$ n'est pas orthogonal\\
\newpage
\noindent $\bullet$ Soit $P_0 = (X - 1)(X - 2), P_1 = X(X - 2), P_0 = X(X - 1)$\\
$f(P_0, P_1) = P_0(0)P_1(0) + P_0(1)P_1(1) + P_0(2)P_1(2) = 0$\\
$f(P_0, P_2) = P_0(0)P_2(0) + P_0(1)P_2(1) + P_0(2)P_2(2) = 0$\\
$f(P_1, P_2) = P_1(0)P_2(0) + P_1(1)P_2(1) + P_1(2)P_2(2) = 0$\\
Donc la famille $\{P_0, P_1, P_2\}$ est orthogonal\\
$||P_0|| = \sqrt{f(P_0, P_0)} = \sqrt{4} = 2$\\
$||P_1|| = \sqrt{f(P_1, P_1)} = \sqrt{1} = 1$\\
$||P_2|| = \sqrt{f(P_2, P_2)} = \sqrt{4} = 2$\\
Donc la famille $\{\frac{P_0}{2}, P_1, \frac{P_2}{2}\}$ est orthonormée\\
\end{document}
\documentclass{article}
\usepackage[utf8]{inputenc}
\usepackage{amsfonts}
\usepackage{amsmath}
\usepackage{graphicx}
\usepackage[a4paper, total={6in, 8in}]{geometry}
\usepackage{setspace}
\usepackage{xcolor}

\everymath{\displaystyle}

\newcommand\tab[1][1cm]{\hspace*{#1}}
\doublespacing
\author{Frederic Becerril}

\makeatletter
\renewcommand*\env@matrix[1][*\c@MaxMatrixCols c]{%
  \hskip -\arraycolsep
  \let\@ifnextchar\new@ifnextchar
  \array{#1}}
\makeatother

\def\env@matrix{\hskip -\arraycolsep
  \let\@ifnextchar\new@ifnextchar
  \array{*\c@MaxMatrixCols c}}


\newcommand{\scalaire}[2]{\left< #1 | #2\right>}
\newcommand{\important}[1]{{\color{red}\underline{\textbf{#1}}}}
\newcommand{\hbrace}[2]{\underset{#1}{\underbrace{#2}}}
\newcommand{\mylim}[2]{\underset{#1 \rightarrow #2}{\longrightarrow}}
\newcommand{\mysim}[2]{\underset{#1 \rightarrow #2}{\sim}}

\begin{document}

\part*{Exerice 1}

On sait que la symétrie orthogonal est égal a $s = 2p - Id$\\
$p(\begin{pmatrix}
    1\\
    0\\
\end{pmatrix}) = \frac{\scalaire{\begin{pmatrix}
    1\\
    0\\
\end{pmatrix}}{\begin{pmatrix}
    \cos \alpha\\
    \sin \alpha
\end{pmatrix}}}{||\begin{pmatrix}
    \cos \alpha\\
    \sin \alpha\\
\end{pmatrix}||} \begin{pmatrix}
    \cos \alpha\\
    \sin \alpha\\
\end{pmatrix} = \begin{pmatrix}
    \cos \alpha * \cos \alpha\\
    \cos \alpha * \sin \alpha\\
\end{pmatrix}$\\
$p(\begin{pmatrix}
    0\\
    1\\
\end{pmatrix}) = \frac{\scalaire{\begin{pmatrix}
    0\\
    1\\
\end{pmatrix}}{\begin{pmatrix}
    \cos \alpha\\
    \sin \alpha
\end{pmatrix}}}{||\begin{pmatrix}
    \cos \alpha\\
    \sin \alpha\\
\end{pmatrix}||} \begin{pmatrix}
    \cos \alpha\\
    \sin \alpha\\
\end{pmatrix} = \begin{pmatrix}
    \sin \alpha * \cos \alpha\\
    \sin \alpha * \sin \alpha\\
\end{pmatrix}$\\
\\
$P = \begin{pmatrix}
    \cos^2 \alpha & \cos \alpha \sin \alpha\\
    \cos \alpha \sin \alpha & \sin^2 \alpha\\
\end{pmatrix}$ \vspace{3mm} \\
$S = 2P - I_2 = 2\begin{pmatrix}
    \cos^2 \alpha & \cos \alpha \sin \alpha\\
    \cos \alpha \sin \alpha & \sin^2 \alpha\\
\end{pmatrix} - \begin{pmatrix}
    1 & 0\\
    0 & 1\\
\end{pmatrix} = \begin{pmatrix}
    2 \cos^2 \alpha - 1 & 2 \cos \alpha \sin \alpha\\
    2 \cos \alpha \sin \alpha & 2 \sin^2 \alpha - 1\\
\end{pmatrix}$\\
$\tab[2.1cm]= \begin{pmatrix}
    \cos(2 \alpha) & \sin(2 \alpha)\\
    \sin(2 \alpha) & -\cos(2 \alpha)\\
\end{pmatrix}$\\
\\
2) $S_\alpha S_\beta = \begin{pmatrix}
    \cos(2 \alpha) & \sin(2 \alpha)\\
    \sin(2 \alpha) & -\cos(2 \alpha)\\
\end{pmatrix} \begin{pmatrix}
    \cos(2 \beta) & \sin(2 \beta)\\
    \sin(2 \beta) & -\cos(2 \beta)\\
\end{pmatrix} $\\
$\tab[1.35cm] = \begin{pmatrix}
    \cos(2 \alpha) \cos(2 \beta) + \sin(2 \alpha) \sin(2 \beta) & \cos(2 \alpha) \sin(2 \beta) - \sin(2 \alpha) \cos(2 \beta)\\
    \sin(2 \alpha) \cos(2 \beta) - \cos(2 \alpha) \sin(2 \beta) & \sin(2 \alpha) \sin(2 \beta) + \cos(2 \alpha) \cos(2 \beta) \\
\end{pmatrix}$\\
$\tab[1.35cm] = \begin{pmatrix}
    \cos(2 \alpha - 2 \beta) & -\sin(2 \alpha - 2 \beta)\\
    \sin(2 \alpha - 2 \beta) & \cos(2 \alpha - 2 \beta) \\
\end{pmatrix}$\\
Matrice de rotation d'angle $2 (\alpha - \beta) = R_{2(\alpha - \beta)}$\\
$S_\beta S_\alpha$ = Matrice de rotation d'angle $2 (\beta - \alpha) = R_{2(\beta - \alpha)}$
\newpage
\noindent 3) Soit un angle $\theta$ on prend $\alpha = \frac{1}{2} \theta$ et $\beta = 0$ on a bien que $2(\alpha - \beta) = \theta$\\
Donc la rotation d'angle $\theta$ peut être écris comme $R_{\theta} = S_{\frac{\theta}{2}} S_{0}$\\
4) Non le groupe O(P) n'est pas commutatif car $S_{0} S_{\frac{\pi}{4}} = R_{\frac{\pi}{2}} \neq R_{-\frac{\pi}{2}} = S_{\frac{\pi}{4}} S_{0}$
\end{document}
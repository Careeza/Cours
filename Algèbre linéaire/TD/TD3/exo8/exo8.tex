\documentclass{article}
\usepackage[utf8]{inputenc}
\usepackage{amsfonts}
\usepackage{amsmath}
\usepackage{graphicx}
\usepackage[a4paper, total={6in, 8in}]{geometry}
\usepackage{setspace}
\usepackage{xcolor}

\everymath{\displaystyle}

\newcommand\tab[1][1cm]{\hspace*{#1}}
\doublespacing
\author{Frederic Becerril}

\newcommand{\scalaire}[2]{\left< #1 | #2\right>}
\newcommand{\important}[1]{{\color{red}\underline{\textbf{#1}}}}
\newcommand{\hbrace}[2]{\underset{#1}{\underbrace{#2}}}
\newcommand{\mylim}[2]{\underset{#1 \rightarrow #2}{\longrightarrow}}
\newcommand{\mysim}[2]{\underset{#1 \rightarrow #2}{\sim}}

\begin{document}

\part*{Exerice 8}

E espace euclidien\\
F - sev  de E\\
$s \circ s(x) = s(s(x))$\\
$\tab[1.25cm] =s(2P_F(x) - x)$\\
$\tab[1.25cm] =2s(P_F(x)) - s(x)$\\
or $P_F(x) \in F$ donc $s(P_F(x)) = P_F(x)$\\
$s \circ s(x) = 2P_F(x) - s(x)$\\
$\tab[1.25cm] = 2P_F(x) - (2P_F(x) - x)$\\
$\tab[1.25cm] = x$\\
On a bien $s \circ s = Id_3$\\
Comme s est une isométrie, il existe une base orthonormée dans laquelle la matrice associée S est orthogonale\\
Càd $SS^T = S^TS = I$\\
Comme $s \circ s = Id$, on a $S^2 = I$\\
En conclusion $S = S^T$, càd S-symétrique\\
Soit s-isométrie dans une base ON la matrice M de s est orthogonale et symétrique. Comme M est 
symétrique, elle est diagonalisable, donc par Ex 3.6, s'est une symétrie orthogonale\\
\underline{Remarque} Si u-isométrie, alors u est une symétrie orthogonale $\Leftrightarrow$ $u \circ u = Id$

\end{document}
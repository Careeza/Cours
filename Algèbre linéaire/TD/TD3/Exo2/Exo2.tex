\documentclass{article}
\usepackage[utf8]{inputenc}
\usepackage{amsfonts}
\usepackage{amsmath}
\usepackage{graphicx}
\usepackage[a4paper, total={6in, 8in}]{geometry}
\usepackage{setspace}

\newcommand\tab[1][1cm]{\hspace*{#1}}
\doublespacing
\author{Frederic Becerril}

\makeatletter
\renewcommand*\env@matrix[1][*\c@MaxMatrixCols c]{%
  \hskip -\arraycolsep
  \let\@ifnextchar\new@ifnextchar
  \array{#1}}
\makeatother

\def\env@matrix{\hskip -\arraycolsep
  \let\@ifnextchar\new@ifnextchar
  \array{*\c@MaxMatrixCols c}}

\begin{document}

\part*{Exerice 2}

$\bullet$ Soit $M = \begin{pmatrix}
    a & b & c\\
    0 & d & e\\
    0 & 0 & f\\
\end{pmatrix}, \tab det M = a * d * f$ \vspace*{5mm} \\
Donc si M est inversible alors $a, d, f \neq 0$ \vspace*{5mm}\\
$\begin{pmatrix}[ccc|ccc]
    a & b & c & 1 & 0 & 0\\
    0 & d & e & 0 & 1 & 0\\
    0 & 0 & f & 0 & 0 & 1\\
\end{pmatrix} \rightarrow \begin{pmatrix}[ccc|ccc]
    a & b & 0 & 1 & 0 & -\frac{c}{f}\\
    0 & d & 0 & 0 & 1 & -\frac{e}{f}\\
    0 & 0 & 1 & 0 & 0 & \frac{1}{f}\\
\end{pmatrix} \rightarrow \begin{pmatrix}[ccc|ccc]
    a & 0 & 0 & 1 & -\frac{b}{d} & -\frac{c}{f} + \frac{be}{fd}\\
    0 & 1 & 0 & 0 & \frac{1}{d} & -\frac{e}{fd}\\
    0 & 0 & 1 & 0 & 0 & \frac{1}{f}\\
\end{pmatrix}\vspace*{5mm} \\ \rightarrow \begin{pmatrix}[ccc|ccc]
    1 & 0 & 0 & \frac{1}{a} & -\frac{b}{da} & \frac{be - cd}{fda}\\
    0 & 1 & 0 & 0 & \frac{1}{d} & -\frac{e}{fd}\\
    0 & 0 & 1 & 0 & 0 & \frac{1}{f}\\
\end{pmatrix}$ \vspace*{5mm}\\
Donc $M^{-1} = \begin{pmatrix}
    \frac{1}{a} & -\frac{b}{da} & \frac{be - cd}{fda}\\
    0 & \frac{1}{d} & -\frac{e}{fd}\\
    0 & 0 & \frac{1}{f}\\
\end{pmatrix}$\\
$\bullet$ Soit $u_1 = \begin{pmatrix}
    a\\
    0\\
    0\\
\end{pmatrix}, \; u_2 = \begin{pmatrix}
    b\\
    d\\
    0\\
\end{pmatrix} \; u_3 = \begin{pmatrix}
    c\\
    e\\
    f\\
\end{pmatrix}$
Pour que $M \in O_3(\mathbb{R})$ il faut:
\begin{itemize}
    \item $||u_1|| = ||u_2|| = ||u_3|| = 1 \Leftrightarrow |u_1||^2 = ||u_2||^2 = ||u_3||^2 = 1$
    \item $<u_1 | u_2> = 0$
    \item $<u_1 | u_3> = 0$
    \item $<u_2 | u_3> = 0$
\end{itemize}
$||u_1||^2 = <u_1|u_1> = a^2$\\
Donc $||u_1||^2 = 1 \Leftrightarrow a = 1 \mbox{ ou } a = -1$\\
\newpage
\noindent \underline{Cherchons b et d} \vspace*{5mm} \\ 
$\left\{
    \begin{array}{ll}
        b^2 + d^2 = 1\\
        a * b = 0\\
    \end{array}
\right. \Leftrightarrow \left\{
    \begin{array}{ll}
        d^2 = 1\\
        b = 0 \mbox{ *}\\
    \end{array}
\right. \Leftrightarrow \left\{
    \begin{array}{ll}
        d = 1 \mbox{ ou } d = -1\\
        b = 0 \mbox{ *}\\
    \end{array}
\right.$ \vspace*{5mm}\\
* Car $a \neq 0$\\
On a pour l'instant:\\
$M = \begin{pmatrix}
    \pm 1 & 0 & c\\
    0 & \pm 1 & e\\
    0 & 0 & f\\
\end{pmatrix}$\\
\underline{Cherchons c, e et f} \vspace*{5mm} \\
$\left\{
    \begin{array}{ll}
        c^2 + d^2 + f^2 = 0\\
        a * c = 0\\
        b * e = 0\\
    \end{array}
\right. \Leftrightarrow \left\{
    \begin{array}{ll}
        f = 1 \mbox{ ou } f = -1\\
        c = 0\\
        e = 0\\
    \end{array}
\right.$ \vspace*{5mm}\\
Donc toute les matrices triangulaire sup dans $O_3(\mathbb{R})$ sont:\vspace*{5mm} \\
$\begin{pmatrix}
    1 & 0 & 0\\
    0 & 1 & 0\\
    0 & 0 & 1\\
\end{pmatrix}, \begin{pmatrix}
    1 & 0 & 0\\
    0 & 1 & 0\\
    0 & 0 & -1\\
\end{pmatrix}, \begin{pmatrix}
    1 & 0 & 0\\
    0 & -1 & 0\\
    0 & 0 & 1\\
\end{pmatrix}, \begin{pmatrix}
    1 & 0 & 0\\
    0 & -1 & 0\\
    0 & 0 & -1\\
\end{pmatrix}, \vspace*{5mm} \\ 
\begin{pmatrix}
    -1 & 0 & 0\\
    0 & 1 & 0\\
    0 & 0 & 1\\
\end{pmatrix}, \begin{pmatrix}
    -1 & 0 & 0\\
    0 & 1 & 0\\
    0 & 0 & -1\\
\end{pmatrix}, \begin{pmatrix}
    -1 & 0 & 0\\
    0 & -1 & 0\\
    0 & 0 & 1\\
\end{pmatrix}, \begin{pmatrix}
    -1 & 0 & 0\\
    0 & -1 & 0\\
    0 & 0 & -1\\
\end{pmatrix}$\\

\newpage

\noindent $\bullet$ Soit  $M = \begin{pmatrix}
    a & b & c\\
    d & e & f\\
    g & h & i\\
\end{pmatrix}$\\
Soit $v_1 = \begin{pmatrix}
    a\\
    d\\
    g\\
\end{pmatrix}, v_2 = \begin{pmatrix}
    b\\
    e\\
    h\\
\end{pmatrix}, v_3 = \begin{pmatrix}
    c\\
    f\\
    i\\
\end{pmatrix}$\\
Et $w_1 = \begin{pmatrix}
    a & b & c
\end{pmatrix}, w_2 = \begin{pmatrix}
    d & e & f
\end{pmatrix}, w_3 = \begin{pmatrix}
    g & h & i
\end{pmatrix}$\\
Comme $M \in O_3(\mathbb{R})$ on a:\\
$v_1 \circ v_2 = v_1 \circ v_3 = v_2 \circ v_3 = 0$\\ 
$w_1 \circ w_2 = w_1 \circ w_3 = w_2 \circ w_3 = 0$\\
\underline{cas 1} : $a \neq 0$ on a directement\\
$M = \begin{pmatrix}
    a & 0 & 0\\
    0 & e & f\\
    0 & h & i\\
\end{pmatrix}$\\
Puis comme $||v_1|| = 1$ on a, $a = 1$\\
\underline{cas 1.1} : $e \neq 0$ on a directement\\
$M = \begin{pmatrix}
    1 & 0 & 0\\
    0 & e & 0\\
    0 & 0 & i\\
\end{pmatrix}$\\
Puis comme $||v_2|| = ||v_3|| = 1$ on a $e = 1$ et $i = 1$\\
$M = \begin{pmatrix}
    1 & 0 & 0\\
    0 & 1 & 0\\
    0 & 0 & 1\\
\end{pmatrix}$\\
En suivant le même raisonnement on obtient que si $M \in O_3(\mathbb{R})$ il y exactement a 1 sur chaque ligne et sur chaque colonne\\
Exemple d'une autre matrice $M \in O_3(\mathbb{R})$\\
$M = \begin{pmatrix}
    0 & 1 & 0\\
    1 & 0 & 0\\
    0 & 0 & 1\\
\end{pmatrix}$

\end{document}
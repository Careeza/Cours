\documentclass{article}
\usepackage[utf8]{inputenc}
\usepackage{amsfonts}
\usepackage{amsmath}
\usepackage{graphicx}
\usepackage[a4paper, total={6in, 8in}]{geometry}
\usepackage{setspace}
\usepackage{xcolor}

\everymath{\displaystyle}

\newcommand\tab[1][1cm]{\hspace*{#1}}
\doublespacing
\author{Frederic Becerril}

\newcommand{\scalaire}[2]{\left< #1 | #2\right>}
\newcommand{\important}[1]{{\color{red}\underline{\textbf{#1}}}}
\newcommand{\hbrace}[2]{\underset{#1}{\underbrace{#2}}}
\newcommand{\mylim}[2]{\underset{#1 \rightarrow #2}{\longrightarrow}}
\newcommand{\mysim}[2]{\underset{#1 \rightarrow #2}{\sim}}
% \newcommand{\citer}[2]{\og #2 \fg{} (#1)}
% \color{red}

\begin{document}

\part*{Exerice 5}

On étudie $ker(u - Id_E)$ et $Im(u - Id_E)$\\
soit $x \in ker(u - Id_E)$ et $y \in Im(u - Id_E)$\\
On va étudier $<x|y>$\\
$x \in ker(u - Id_E) \Rightarrow (u - Id_E)(x) = 0$\\
$\tab[2.8cm] \Rightarrow u(x) - x = 0$\\
$\tab[2.8cm] \Rightarrow u(x) = x$\\
$y \in Im(u - Id_E) \Rightarrow \exists x' \in (u - Id_E)$ tq $(u - Id_E)x' = y$\\
$\tab[2.7cm] \Rightarrow u(x') - x' = y$\\
% $\tab[2.7cm] \Rightarrow u(x) = x$\\
On a $\scalaire{x}{y} = \scalaire{x}{u(x') - x'}$\\
$\tab[1.75cm] = \scalaire{x}{u(x')} - \scalaire{x}{x'}$\\
$\tab[1.75cm] = \scalaire{u(x)}{u(x')} - \scalaire{x}{x'}$\\
Or comme u est une isométrie le produit scalaire reste inchangé.\\
$\tab[1.75cm] = \scalaire{x}{x'} - \scalaire{x}{x'}$\\
$\tab[1.75cm] = 0$\\
Donc $ker(u - Id_E)$ et $Im(u - Id_E)$ sont orthogonaux\\
Comme $ker(u - Id_E)$ et $Im(u - Id_E)$ sont orthogonaux on a que:\\
$ker(u - Id_E)$ et $Im(u - Id_E)$ sont libres\\
$ker(u - Id_E) \cap Im(u - Id_E) = \{0\}$\\
Et on a que $dim(U - Id_E) = dim(ker(u - Id_E)) + dim(Im(u - Id_E))$\\
Or $dim(U - Id_E) = dim(E)$, alors $dim(ker(u - Id_E)) + dim(Im(u - Id_E)) = dim(E)$\\
Donc $ker(u - Id_E) \oplus Im(u - Id_E) = E$\\

\end{document}
\documentclass{article}
\usepackage[utf8]{inputenc}
\usepackage{amsfonts}
\usepackage{amsmath}
\usepackage{graphicx}
\usepackage[a4paper, total={6in, 8in}]{geometry}
\usepackage{setspace}
\usepackage{enumitem}
\usepackage{mathabx}
\usepackage{stmaryrd}


\newcommand\tab[1][1cm]{\hspace*{#1}}
\doublespacing
\author{Frederic Becerril}

\makeatletter
\renewcommand*\env@matrix[1][*\c@MaxMatrixCols c]{%
  \hskip -\arraycolsep
  \let\@ifnextchar\new@ifnextchar
  \array{#1}}
\makeatother

\def\env@matrix{\hskip -\arraycolsep
  \let\@ifnextchar\new@ifnextchar
  \array{*\c@MaxMatrixCols c}}

\begin{document}

\part*{Exerice 3}

$\bullet$ Soit $X, Y \in \mathbb{R}^n$ et $P \in O_3(\mathbb{R})$\\
$<PX | PY> = (PX)^T \cdot PY = X^T P^T \cdot P Y$\\
Or comme $P \in O_3(\mathbb{R})$ $P^T \cdot P = I_3$\\
Donc $<PX | PY> = X^T Y = <X|Y>$\\
Comme $<PX | PY> = <X|Y>$\\
on a $||PX|| = \sqrt{<PX|PX>} = \sqrt{<X|X>} = ||X||$\\
Si P admet une valeur propre $\lambda \in \mathbb{R}$\\
Soit U le vecteur propre associé a cette valeur propre\\
On a d'une part $||PU|| = ||\lambda U|| = |\lambda| \times ||U||$ car U vecteur propre\\
D'autre part on a $||PU|| = ||U||$ car P est une isométrie\\
Donc $||U|| = |\lambda| \times ||U|| \Leftrightarrow \lambda \pm 1$\\
$\bullet$ Soit $Z \in \mathbb{C}^n$ un vecteur propre associé a la valeur propre $\lambda \in \mathbb{C}$\\
$Z = \begin{pmatrix}
    z_1\\
    \vdots\\
    z_n\\
\end{pmatrix} = \begin{pmatrix}
    a_1 + i b_1\\
    \vdots\\
    a_n + i b_n\\
\end{pmatrix}$ \\
$\overline{Z} = \begin{pmatrix}
    \overline{z_1}\\
    \vdots\\
    \overline{z_n}\\
\end{pmatrix} = \begin{pmatrix}
    a_1 - i b_1\\
    \vdots\\
    a_n - i b_n\\
\end{pmatrix}$ \\

\newpage

\noindent \begin{enumerate}[label=\Alph*)]
    \item $\overline{Z}^T \circ Z = \sum_{i=1}^n z_i \times \overline{z_i} = \sum_{i=1}^n (a_i)^2 + (b_i)^2$ 
    \item $(P\overline{Z})^T \circ (PZ) = \overline{Z}^TP^T \circ PZ? = \overline{Z}^T Z$
    \item Soit $f: \mathbb{C}^n \rightarrow \mathbb{R}\\
    \tab Z \longmapsto \overline{Z}^T \circ Z$\\
    Soit $Z \in \mathbb{C}^n$ et $\lambda \in \mathbb{C}$, montrons que $f(\lambda Z) = |\lambda|^2f(Z)$\\
    $f(\lambda Z) = \overline{\lambda Z}^T \circ \lambda Z = (\overline{\lambda} \times \overline{Z})^T \circ (\lambda \times) Z$\\
    $\tab = \overline{\lambda} \times \lambda \times (\overline{Z}^T Z) = |\lambda|^2 f(Z)$\\
    On calcule de deux manière différente $f(PZ)$:
    \begin{itemize}
        \item $f(PZ) = f(Z)$ vu a la question 2
        \item $f(PZ) = f(\lambda Z) = |\lambda|^2 f(Z)$
    \end{itemize}
    Donc $|\lambda|^2 f(Z) = 1 * f(Z) \Leftrightarrow |\lambda|^2 = 1$\\
    Or le module d'un nombre complexe est toujours positif\\
    Donc on a $|\lambda| = 1$
\end{enumerate}

% \newpage
% \noindent Soit $f: \mathbb{C}^n \rightarrow \mathbb{R}$\\
% $\tab[1.5cm] Z \longmapsto \sqrt{\overline{Z}^T \circ Z}$\\
% Montrons que f est une norme de $\mathbb{C}^n$\\
% \begin{itemize}
%     \item Soit $\alpha \in \mathbb{R} \mbox{ et } Z \in \mathbb{C}^n$\\
%     $f(\alpha Z) = \sqrt{\overline{\alpha Z}^T \circ \alpha Z} = \sqrt{\sum_{i=1}^n (\alpha a_i)^2 + (\alpha b_i)^2} = \sqrt{\sum_{i=1}^n \alpha^2 ((a_i)^2 + (b_i)^2)}$\\
%     $\tab = \sqrt{\alpha^2 \sum_{i=1}^n (a_i)^2 + (b_i)^2} = |a| f(Z)$
%     \item Soit $Z \in \mathbb{C}^n$\\
%     $f(Z) = \sqrt{\sum_{i=1}^n (a_i)^2 + (b_i)^2} \geq 0$\\
%     Si $f(Z) = 0 \Leftrightarrow \sqrt{\sum_{i=1}^n (a_i)^2 + (b_i)^2} = 0 \Leftrightarrow \sum_{i=1}^n (a_i)^2 + (b_i)^2 = 0\\
%     \Leftrightarrow \forall i \in \llbracket1, n\rrbracket,\; a_i = b_i = 0$
%     \item Soit $Z, Z' \in \mathbb{C}^n$\\
%     $f(Z + Z') = \sqrt{\sum_{i=1}^n (a_i + a'_i)^2 + (b_i + b'_i)^2}\\
%     \tab[1.5cm] = \sqrt{\sum_{i=1}^n (a_i)^2 + 2 a_i a'_i + (a'_i)^2 + (b_i)^2 + 2 b_i b'_i + (b'_i)^2}\\
%     \tab[1.5cm] = \sqrt{\sum_{i=1}^n (a_i)^2 + (b_i)^2 + \sum_{i=1}^n (a'_i)^2 + (b'_i)^2 + \sum_{i=1}^n 2(a_i a'_i + b_i b'_i)}$
% \end{itemize}

\end{document}
\documentclass{article}
\usepackage[utf8]{inputenc}
\usepackage{amsfonts}
\usepackage{amsmath}
\usepackage{graphicx}
\usepackage[a4paper, total={6in, 8in}]{geometry}
\usepackage{setspace}
\usepackage{xcolor}

\everymath{\displaystyle}

\newcommand\tab[1][1cm]{\hspace*{#1}}
\doublespacing
\author{Frederic Becerril}

\makeatletter
\renewcommand*\env@matrix[1][*\c@MaxMatrixCols c]{%
  \hskip -\arraycolsep
  \let\@ifnextchar\new@ifnextchar
  \array{#1}}
\makeatother

\def\env@matrix{\hskip -\arraycolsep
  \let\@ifnextchar\new@ifnextchar
  \array{*\c@MaxMatrixCols c}}


\newcommand{\scalaire}[2]{\left< #1 | #2\right>}
\newcommand{\important}[1]{{\color{red}\underline{\textbf{#1}}}}
\newcommand{\hbrace}[2]{\underset{#1}{\underbrace{#2}}}
\newcommand{\mylim}[2]{\underset{#1 \rightarrow #2}{\longrightarrow}}
\newcommand{\mysim}[2]{\underset{#1 \rightarrow #2}{\sim}}

\begin{document}

\part*{Exerice 7}

$Mat_B(P) = \begin{pmatrix}
    1 & 0 & 0\\
    0 & -1 & 0\\
    0 & 0 & -1\\
\end{pmatrix}$
Ou $B = \left\{\begin{pmatrix}
    1\\
    1\\
    0\\
\end{pmatrix}, \left\{\begin{pmatrix}
    1\\
    1\\
    0\\
\end{pmatrix}
\right\}^\perp\right\}$\\
$Vect\left\{\begin{pmatrix}
    1\\
    1\\
    0\\
\end{pmatrix}\right\}^\perp = \{(x, y, x) \in \mathbb{R}^3 | x + y = 0\} = \left\{
    \begin{array}{ll}
        x = -y\\
        y \in \mathbb{R}\\
        z \in \mathbb{R}\\        
    \end{array}
    \right.$\\
$Vect\left\{\begin{pmatrix}
    1\\
    1\\
    0\\
\end{pmatrix}\right\}^\perp = \begin{pmatrix}
    -y\\
    y\\
    z\\
\end{pmatrix} = y\begin{pmatrix}
    -1\\
    1\\
    0\\
\end{pmatrix} + z \begin{pmatrix}
    0\\
    0\\
    1\\
\end{pmatrix} = Vect\left\{\begin{pmatrix}
    -1\\
    1\\
    0\\
\end{pmatrix}, \begin{pmatrix}
    0\\
    0\\
    1\\
\end{pmatrix}\right\}$\\
Donc B = $Vect\left\{\begin{pmatrix}
    1\\
    1\\
    0\\
\end{pmatrix},\begin{pmatrix}
    -1\\
    1\\
    0\\
\end{pmatrix}, \begin{pmatrix}
    0\\
    0\\
    1\\
\end{pmatrix}\right\}$ \vspace{2mm}\\
Soit la matrice P' de passage de $B$ à $B_3$\\
$P' = \begin{pmatrix}
    1 & -1 & 0\\
    1 & 1 & 0\\
    0 & 0 & 1\\
\end{pmatrix}$\\
$\begin{pmatrix}[ccc|ccc]
    1 & -1 & 0 & 1 & 0 & 0\\
    1 & 1 & 0 & 0 & 1 & 0\\
    0 & 0 & 1 & 0 & 0 & 1\\
\end{pmatrix} \rightarrow \begin{pmatrix}[ccc|ccc]
    1 & -1 & 0 & 1 & 0 & 0\\
    0 & 2 & 0 & -1 & 1 & 0\\
    0 & 0 & 1 & 0 & 0 & 1\\
\end{pmatrix} \rightarrow \begin{pmatrix}[ccc|ccc]
    1 & -1 & 0 & 1 & 0 & 0\\
    0 & 1 & 0 & -\frac{1}{2} & \frac{1}{2} & 0\\
    0 & 0 & 1 & 0 & 0 & 1\\
\end{pmatrix}$\vspace{3mm}\\
$ \rightarrow \begin{pmatrix}[ccc|ccc]
    1 & 0 & 0 & \frac{1}{2} & \frac{1}{2} & 0\\
    0 & 1 & 0 & -\frac{1}{2} & \frac{1}{2} & 0\\
    0 & 0 & 1 & 0 & 0 & 1\\
\end{pmatrix}$\\
$(P')^-1 = \begin{pmatrix}
    \frac{1}{2} & \frac{1}{2} & 0\\
    -\frac{1}{2} & \frac{1}{2} & 0\\
    0 & 0 & 1\\
\end{pmatrix}$\\
$Mat_{B_3}(P) = P'\begin{pmatrix}
    1 & 0 & 0\\
    0 & -1 & 0\\
    0 & 0 & -1\\
\end{pmatrix}(P')^{-1}$ \vspace{3mm}\\
$\tab[1.7cm] = \begin{pmatrix}
    1 & -1 & 0\\
    1 & 1 & 0\\
    0 & 0 & 1\\
\end{pmatrix}\begin{pmatrix}
    1 & 0 & 0\\
    0 & -1 & 0\\
    0 & 0 & -1\\
\end{pmatrix}\begin{pmatrix}
    \frac{1}{2} & \frac{1}{2} & 0\\
    -\frac{1}{2} & \frac{1}{2} & 0\\
    0 & 0 & 1\\
\end{pmatrix}$\\
$\tab[1.7cm] = \begin{pmatrix}
    1 & 1 & 0\\
    1 & -1 & 0\\
    0 & 0 & -1\\
\end{pmatrix}\begin{pmatrix}
    \frac{1}{2} & \frac{1}{2} & 0\\
    -\frac{1}{2} & \frac{1}{2} & 0\\
    0 & 0 & 1\\
\end{pmatrix}$\\
$\tab[1.7cm] = \begin{pmatrix}
    0 & 1 & 0\\
    1 & 0 & 0\\
    0 & 0 & -1\\
\end{pmatrix}$
\newpage

\paragraph{\underline{Méthode 2}:} $Vect \{(1, 1, 0)^T\} = D$\\
$S_D = 2P_D - Id_3$\\
$P_D(x) = \frac{\scalaire{x}{(1, 1, 0)^T}}{\scalaire{(1, 1, 0)^T}{(1, 1, 0)^T}} \begin{pmatrix}
    1\\
    1\\
    0\\
\end{pmatrix}$\\
$S_D\begin{pmatrix}
    1\\
    0\\
    0\\
\end{pmatrix} = 2P_D\begin{pmatrix}
    1\\
    0\\
    0\\
\end{pmatrix} - \begin{pmatrix}
    1\\
    0\\
    0\\
\end{pmatrix} = 2 \cdot \frac{1}{2} \cdot \begin{pmatrix}
    1\\
    1\\
    0\\
\end{pmatrix} - \begin{pmatrix}
    1\\
    0\\
    0\\
\end{pmatrix} = \begin{pmatrix}
    0\\
    1\\
    0\\
\end{pmatrix}$\\
$S_D\begin{pmatrix}
    0\\
    1\\
    0\\
\end{pmatrix} = 2 \cdot \frac{1}{2} \cdot \begin{pmatrix}
    1\\
    1\\
    0\\
\end{pmatrix} - \begin{pmatrix}
    0\\
    1\\
    0\\
\end{pmatrix} = \begin{pmatrix}
    1\\
    0\\
    0\\
\end{pmatrix}$\\
$S_D\begin{pmatrix}
    0\\
    0\\
    1\\
\end{pmatrix} = 2 \cdot 0 \cdot \begin{pmatrix}
    1\\
    1\\
    0\\
\end{pmatrix} - \begin{pmatrix}
    0\\
    0\\
    1\\
\end{pmatrix} = \begin{pmatrix}
    0\\
    0\\
    -1\\
\end{pmatrix}$\\
Donc $S = \begin{pmatrix}
    0 & 1 & 0\\
    1 & 0 & 0\\
    0 & 0 & -1\\
\end{pmatrix}$\\
\\
2- Soit le plan d'eq $x - y + z = 0$\\
On a alors $v = (1, -1, 1)$ un vecteur orthogonal au plan\\
Soit F = $Vect \{(1, -1, 1)\}$\\
On a $Q_F = 2 P_F - Id_3$\\
$P_F = \frac{\scalaire{\begin{pmatrix}
    x\\
    y\\
    z\\
\end{pmatrix}}{\begin{pmatrix}
    1\\
    -1\\
    1\\
\end{pmatrix}}}{\left|\left|\begin{pmatrix}
    1\\
    -1\\
    1
\end{pmatrix}\right|\right|^2}\begin{pmatrix}
    1\\
    -1\\
    1\\
\end{pmatrix} = \frac{x - y + z}{3}\begin{pmatrix}
    1\\
    -1\\
    1\\
\end{pmatrix} = \begin{pmatrix}
    \frac{x}{3} -\frac{y}{3} + \frac{z}{3}\\ 
    -\frac{x}{3} +\frac{y}{3} - \frac{z}{3}\\ 
    \frac{x}{3} -\frac{y}{3} + \frac{z}{3}\\ 
\end{pmatrix}$\\
On en déduit la matrice $P_F = \frac{1}{3}\begin{pmatrix}
    1 & -1 & 1\\
    -1 & 1 & -1\\
    1 & -1 & 1\\
\end{pmatrix}$\\
$Q = \begin{pmatrix}
    1 & 0 & 0\\
    0 & 1 & 0\\
    0 & 0 & 1\\
\end{pmatrix} - \frac{2}{3} \begin{pmatrix}
    1 & -1 & 1\\
    -1 & 1 & -1\\
    1 & -1 & 1\\
\end{pmatrix} = \frac{1}{3} \begin{pmatrix}
    1 & 2 & -2\\
    2 & 1 & 2\\
    -2 & 2 & 1\\
\end{pmatrix}$\vspace{3mm}\\
3- Une matrice M est orthogonale si $MM^T = M^TM = Id$\\
On calcule $SS^T = \begin{pmatrix}
    0 & 1 & 0\\
    1 & 0 & 0\\
    0 & 0 & -1\\
\end{pmatrix}\begin{pmatrix}
    0 & 1 & 0\\
    1 & 0 & 0\\
    0 & 0 & -1\\
\end{pmatrix} = \begin{pmatrix}
    1 & 0 & 0\\
    0 & 1 & 0\\
    0 & 0 & 1\\
\end{pmatrix}$\vspace{3mm}\\
On calcule $QQ^T = \frac{1}{3}\begin{pmatrix}
    1 & 2 & -2\\
    2 & 1 & 2\\
    -2 & 2 & 1\\
\end{pmatrix}\frac{1}{3} \begin{pmatrix}
    -2 & 2 & 1\\
    2 & 1 & 2\\
    1 & 2 & -2\\
\end{pmatrix} = \frac{1}{9} \begin{pmatrix}
    9 & 0 & 0\\
    0 & 9 & 0\\
    0 & 0 & 9\\
\end{pmatrix} =\begin{pmatrix}
    1 & 0 & 0\\
    0 & 1 & 0\\
    0 & 0 & 1\\
\end{pmatrix}$\vspace{3mm}\\
$\lambda$ est une valeur propre de M si $det(M - \lambda I) = 0$\\
On a montré que S est la matrice d'une d'une symétrie orthogonal $\Rightarrow$ donc S diagonalisable:\\
$Sp(s) \subseteq \{-1, 1\}$. Or si $Sp(s) = \{1\}$ ou $Sp(s) = \{-1\}$, cela voudrait dire que S est semblable à $I_3$ ou à $-I_3$ et donc $S = \pm I_3$ ce qui n'est pas le cas\\
Donc \fbox{$Sp(S) = \{-1, 1\}$}\\
De même pour Q \fbox{$Sp(Q) = \{-1, 1\}$}\\

\end{document}
\documentclass{article}
\usepackage[utf8]{inputenc}
\usepackage{amsfonts}
\usepackage{amsmath}
\usepackage{graphicx}
\usepackage[a4paper, total={6in, 8in}]{geometry}
\usepackage{setspace}

\newcommand\tab[1][1cm]{\hspace*{#1}}
\onehalfspacing
\author{Frederic Becerril}

\begin{document}

\part*{Chapitre 1: Produit scalaire et orthogonalité}

\section{Produit scalaire dans un $\mathbb{R}$ev}

\subsection{Orthogonalité des vecteurs}

Comme dans le plan, dans $(E, <|>)$, deux vecteurs x et y sont \underline{orthogonaux} ssi $<x|y> = 0 \tab[0.5cm] x \perp y$ \\
On a de nouveau l'égalité de Pythagore: \\
$x \perp y \Leftrightarrow ||x + y||^2 = ||x||^2 + ||y||^2$

\begin{itemize}
    \item On dit aussi qu'un vecteur x est \underline{unitaire} si $||x|| = 1$
    \item A tout vecteur x non nul on peut bien sûr associer un vecteur unitaire et colinéaire à x en posant: $u = \frac{x}{||x||}$
\end{itemize}

\paragraph{\underline{Définition:}}
Soit $(E, <|>)$ prehilbertien et pour $p \geq 1$ fixé ${f_1, \dots, f_p} = F$ est une famille de vecteurs de E.
\begin{itemize}
    \item La famille F est dite \underline{orthogonale} ssi les vecteurs de la famille sont deux à deux orthogonaux:
$$
\forall 1 \leq i, j \leq p \tab i \neq j \tab <f_i | f_j> = 0
$$
    \item La famille F est dite \underline{orthonormée} si la famille est orthogonale et que tous les vecteurs sont unitaires:
$$
\forall 1 \leq i, j \leq p \tab <f_i | f_j> = 
\left\{
    \begin{array}{ll}
        0 \mbox{ si } i \neq j \\
        1 \mbox{ si } i = j \\
    \end{array}
\right.
$$
\end{itemize}

\paragraph{\underline{Remarque:}} \mbox{}\\
Si $p = 1 \tab F = \{f_1\}$ est orthogonale pour tout $f_1$ de E \\
\tab[2.1cm] $F = \{f_1\}$ est orthonormé pour tout vecteur $f_1$ de E \\

Il y a un lien entre orthogonalité et liberté.

\paragraph{\underline{Proposition:}} Toute famille orthogonale de vecteurs \underline{non nuls} est \underline{libre}. \\
\underline{Attention}
$$
\mbox{Dans } \mathbb{R}^2 \tab[0.3cm]
\left\{
    \begin{array}{ll}
        \begin{pmatrix}
            0\\
            0\\
        \end{pmatrix},
        \begin{pmatrix}
            1\\
            1\\
        \end{pmatrix}
    \end{array}
\right\} \mbox{ est orthogonale mais pas libre}
$$

\paragraph{\underline{Preuve:}} Soit $\{f_1, \dots, f_p\}$ une telle famille, on suppose qu'il existe $(\alpha_1, \dots, \alpha_p) \in \mathbb{R}^p$ tel que:
$$
\alpha_1 f_1 + \dots \alpha_p f_p = 0 = \sum_{i = 1}^p \alpha_i f_1
$$

D'après les Proprietés du produit scalaire, \underline{pour tout} $1 \leq j \leq p$
$$
<0 | f_j> = 0 = <\alpha_1 f_1 + \dots + \alpha_p f_p | f_j>
$$
\tab[6.5cm]= $<\sum_{i = 1}^p \alpha_i f_i | f_j>$ \\
\tab[6.5cm]= $\sum_{i = 1}^p \alpha_i <f_i | f_j> = \alpha_j = 0$ \\
On vient de montrer que: \\
Si $\alpha_1 f_1 + \dots + \alpha_p f_p = 0$ alors $\alpha_1 = \alpha_2 = \dots = \alpha_p = 0$
C'est la définition d'un famille libre.

\paragraph{\underline{Corollaire:}}
\begin{itemize}
    \item Toute famille orthonormé est libre
    \item De toute famille orthogonale de vecteurs non nuls on déduit une famille orthonormé de même taille en normant:
$$
\{f_1, \dots, f_p\} \rightarrow \{\frac{f_1}{||f_1||}, \dots, \frac{f_p}{||f_p||}\}
$$
    \item Sous les mêmes hypothèses sur F $Vect\{f_1, \dots, f_p\}$ est de dimension p.
\end{itemize}

Dans un espace euclidien $(E, <|>)$, a-t-on des bases orthonormées ou orthogonales, i.e. si dans $E = n$\\
A-t-on des familles orthonormées de taille n ? \\
A-t-on des familles orthogonales de vecteurs non nuls de taille n ?.

\paragraph{\underline{Théorème:}} Si $(E, <|>)$ est euclidien (dimension finie) E admet des bases orthogonales, et donc des bases orthonormées (BON)

\newpage

\paragraph{\underline{Preuve:}} On va se concentrer sur l'existence de bases orthogonales, le reste s'en déduisant. \\
Trouver une base orthogonale revient à trouver une famille orthogonale de n vecteurs non nuls, ce qu'on va faire par récurrence sur n \\
Montrons que le résultat est vrai pour tout espace de dimension n par récurrence $(H_n)$
\begin{itemize}
    \item si $n = 1$, on prend un élément non nul de E $f_1$ et $\{f_1\}$ base orthogonale.
    \item Supposons que la Proprieté soit vrai dans tout espace de dim n et on se donne un espace euclidien E de dimension $n + 1$ de produit scalaire $<|>_E$
    \begin{itemize}
        \item[] Soit e une élément non nul de E fixé
        \item[] L'application: $\varphi_e: E \rightarrow \mathbb{R}$
        \item[] $\tab[3cm] x \rightarrow <x|e>_E$ \tab[0.2cm] est bilinéaire.
        \item[] et si $\lambda \in \mathbb{R}, \tab[0.5cm] \varphi_e(\frac{\lambda}{||e||^2}e) = \lambda \tab[0.5cm] \mbox{ donc } Im \varphi_e = \mathbb{R}$
        \item[] donc $F = \mbox{Ker }\varphi_e = \{x \in E \mbox{ tq } <x | e>_E = 0\}$ est un sous-espace vectoriel de dimension n
        \item[] Mais l'application $F \times F \rightarrow R$
        \item[] $\tab[3cm] (x, y) \rightarrow <x|y>_E$ est un produit scalaire sur F
        \item[] Par hypothèses il existe une base orthogonale de F $\{f_1, \dots, f_n\}$
        \item[] $(<f_i|f_j>_E = 0 \mbox{ si } i \neq j)$
        \item[] Mais $e \neq 0$ et $<f_i | e>_E = 0$ 
        \item[] $\{f_1, \dots, f_n, e\}$ est donc une famille orthogonale de E, de n+1 éléments non nuls, donc une base de E 
    \end{itemize}
    \item On a $H_1$ vrai \tab donc $\forall n \geq 1 \; H_n$ vrai \tab $\forall n \in \mathbb{N}^* \; H_n \Rightarrow H_{n+1}$
\end{itemize}

\paragraph{\underline{Corollaire:}} Dans un espace prehilbertien $(E, <|>)$ tout sous espace vectoriel de dimension finie admet une base orthogonale (et un BON)

On va finir ce chapitre en examinant comment un produit scalaire s'exprime dans une base. (Un peu comme le lien entre application linéaire et matrice)

\newpage

\subsection{Espaces euclidien et bases}

\paragraph{\underline{Rappel:}} Si B = $\{e_1, \dots, e_n\}$ est une base de E (dim $E = n$) alors, on a un isomorphisme entre E et $\mathbb{R}^n$ en utilisant les coordonnées
\begin{itemize}
    \item[] $\varphi_B : E \rightarrow \mathbb{R}^n$
    \item[] $x = x_1 e_1 + \dots + x_n + e_n \rightarrow
    \begin{pmatrix}
        x_1 \\
        \vdots \\
        x_n \\
    \end{pmatrix}$
    \item[] $\varphi_B^{-1} : \mathbb{R}^n \rightarrow E$
    \item[] $X = 
    \begin{pmatrix}
        x_1 \\
        \vdots \\
        x_n \\
    \end{pmatrix} \rightarrow x_1 e_1 + \dots + x_n e_n$
\end{itemize}

Si E est euclidien $(<|>)$ alors si $\varphi_B(x) = X =
\begin{pmatrix}
    x_1 \\
    \vdots \\
    x_n \\
\end{pmatrix}, \tab \varphi_B(y) = Y = 
\begin{pmatrix}
    y_1 \\
    \vdots \\
    y_n \\
\end{pmatrix}$ \\
$et <x | y> \;=\; <\sum_{i = 1}^n x_i e_i | \sum_{j = 1}^n y_j e_j > \;=\; \sum_{i = 1}^n\sum_{j = 1}^n x_i y_j <e_i | e_j>$

\paragraph{\underline{Proposition:}}
\begin{itemize}
    \item Soit B une base de E et A la matrice $M_n(\mathbb{R})$ telle que $A_{i,j} = <e_i | e_j>$
    \item Si x a pour coordonnées X dans B et y a pour coordonnées Y dans B $<x|y> = X^T . A . Y$
    \item Par ailleurs $(X, Y) \rightarrow X^T . A . Y$ défini un produit scalaire sur $\mathbb{R}^n \; <|>_A$ \\
tel que $<x|y> = <\varphi_B(x)|\varphi_B(y)>_A$
\end{itemize}

\paragraph{\underline{Remarque:}} $A_{i, j} = A_{j, i}$ soit encore $A^T = A$ (A symétrique)

La preuve est évidente en développant $X^T . A . Y$ et c'est un produit scalaire grâce à l'isomorphisme
$$<X|Y>_A = <\varphi_B^{-1}(x) | \varphi_B^{-1}(Y)>_E$$

\begin{itemize}
    \item Que se passe-t-il si B est orthonormée ?
    \begin{itemize}
        \item[] Si B est orthonormée, pour tous $x, y \in E$ et \\
    $X = \varphi_B(x) \tab[0.5cm] Y = \varphi_B(y) \mbox{ dans } \mathbb{R}^n$\\
    $A = I_n \mbox{ donc } <x|y> = X^T . Y = x_1 y_1 + \dots + x_n y_n$
        \item[] Comme tout espace E euclidien de dimension n admet des bases orthonormées, $(E, <|>)$ est isomorphe à $\mathbb{R}^n$ muni du produit scalaire usuel. Toutes les Proprietés des espaces euclidiens peuvent se ramener aux Proprietés de $(\mathbb{R}^n, \cdot)$
    \end{itemize}
    \item On finit avec 3 remarques
    \begin{itemize}
        \item Sauf contre indication, quand on parle de $\mathbb{R}^n$ muni du produit scalaire usuel
        \item Dans $\mathbb{R}^n$, la base canonique $
        \left\{
            \begin{array}{ll}
                \begin{pmatrix}
                    1\\
                    0\\
                    \vdots \\
                    0 \\
                \end{pmatrix},
                \begin{pmatrix}
                    0\\
                    1\\
                    \vdots \\
                    0 \\
                \end{pmatrix}, \dots,
                \begin{pmatrix}
                    0 \\
                    0 \\
                    \vdots \\
                    1 \\
                \end{pmatrix}
            \end{array}
        \right\}
$ est orthonormée mais ce n'est pas la seul. \\
    Dans $\mathbb{R}^n \; \forall \theta \in \mathbb{R},
    \left\{
        \begin{array}{ll}
            \left(
                \begin{array}{ll}
                    cos(\theta) \\
                    sin(\theta) \\
                \end{array}
            \right)
            \left(
                \begin{array}{ll}
                    -sin(\theta) \\
                    cos(\theta) \\
                \end{array}
            \right)
        \end{array}
    \right\}$ est une BON
    \end{itemize}
    \item On y reviendra plus tard mais si B et B' sont deux bases de $(E, <|>)$ euclidien. et P la matrice de passage de B à B'
    Si (X, Y), (X', Y') sont les coordonnées de x, y respectivement dans B et B'. on a \\
    $<x | y> = X^T . A . Y = X'^T . A' . Y'$ et comme $X' = PX, \; Y' = PY$ \\
    $X'^T . A' . Y' = (PX)^T . A' . (PY) = X^T . P^T . A' . P . Y$ \\
    d'où $A' = P^T . A . P$
\end{itemize}


\end{document}
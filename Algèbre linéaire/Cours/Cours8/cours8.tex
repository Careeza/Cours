\documentclass{article}
\usepackage[utf8]{inputenc}
\usepackage{amsfonts}
\usepackage{amsmath}
\usepackage{graphicx}
\usepackage[a4paper, total={6in, 8in}]{geometry}
\usepackage{setspace}

\newcommand\tab[1][1cm]{\hspace*{#1}}
\doublespacing
\author{Frederic Becerril}

\everymath{\displaystyle}

\begin{document}

\section{Isométrie en dim 2}

E euclidien de dim $E=2$ par exemple

\subsection{Orientation du plan }

Dans le plan une BON $(\vec{i}, \vec{j})$ est directe si l'angle orienté $\widehat{(\vec{i}, \vec{j})} = +\frac{\pi}{2}$ BOND

\paragraph{\underline{Propriété}}:
Si $\mathcal{B}_0 = ((\vec{i}, \vec{j}))$ BOND\\
$\mathcal{B} = (\vec{u}, \vec{v})$ BOND ssi $Mat_{\mathcal{B}_0} (\vec{u}, \vec{v}) \in SO(2)$\\
$\vec{u} = cos \theta \vec{i} + sin \theta \vec{j}$\\
$\vec{v} = -sin \theta \vec{i} + cos \theta \vec{j}$ \vspace*{2mm}\\
P = $\begin{pmatrix}
    cos \theta & -sin \theta\\
    sin \theta & cos \theta\\
\end{pmatrix}$
%include graph
\paragraph{\underline{Définition}}: Si E un espace euclidien. Alors, on oriente l'espace en choisissant une BON $\mathcal{B}_0$ qu'on définit comme directe.\\
Une BON $\mathcal{B}$ est directe (resp. indirecte) si $Mat_{\mathcal{B}_0} B \in SO(n)$ (resp $\in O(n) \backslash SO(n))$\\
$\mathbb{R}^2$ $\mathcal{B}_0 \begin{pmatrix}
    1\\
    0\\
\end{pmatrix}, \begin{pmatrix}
    0\\
    1\\
\end{pmatrix}$\\
$\mathbb{R}^n$ $\mathcal{B}_0 \begin{pmatrix}
    1\\
    0\\
    \vdots\\
    0\\
    0\\
\end{pmatrix}, \begin{pmatrix}
    0\\
    1\\
    0\\
    \vdots\\
    0\\
\end{pmatrix}, \dots \begin{pmatrix}
    0\\
    0\\
    0\\
    \vdots\\
    1\\
\end{pmatrix}$\\
Dans l'espace on verra.

\subsection{\underline{Description de O(2)}}

\paragraph{\underline{Théorème}}: Si $P \in O(2)$, P s'écrit\\
$\begin{pmatrix}
    a & -b\\
    b & a\\
\end{pmatrix}$ avec $a^2 + b^2 = 1$ si $P \in SO(2)$\\
$\begin{pmatrix}
    a & b\\
    b & -a\\
\end{pmatrix}$ avec $a^2 + b^2 = 1$ sinon\\

\paragraph{\underline{Preuve:}}
Si P = $\begin{pmatrix}
    a & c\\
    b & d\\
\end{pmatrix} \in O(2)\\
\tab \Leftrightarrow \left\{
    \begin{array}{ll}
        a^2 + b^2 = 1\\
        c^2 + d^2 = 1\\
        dc + bd = 0\\
    \end{array}    
\right.$\vspace*{3mm}\\
$\tab \Leftrightarrow \exists \alpha, \beta \in \mathbb{R}$\\
$\tab$ tq $a = cos(\alpha)$ \tab $b = sin(\alpha)$\\
$\tab$ tq $c = cos(\beta)$ \tab $b = sin(\beta)$\\
$cos(\alpha)cos(\beta) + sin(\alpha)sin(\beta) = cos(\alpha - \beta) = 0$\\
$\Leftrightarrow$ P = $\begin{pmatrix}
    cos \alpha & cos \beta\\
    sin \alpha & sin \beta\\
\end{pmatrix}$\\ et $\alpha - \beta = \frac{\pi}{2} [\pi]$\\
$\Leftrightarrow$ P = $\begin{pmatrix}
    cos \alpha & cos \beta\\
    sin \alpha & sin \beta\\
\end{pmatrix}$\\ et $\beta = \alpha - \frac{\pi}{2} [\pi]$\\
$\beta = \alpha + \frac{\pi}{2} [2\pi]$ ou $\beta = \alpha - \frac{\pi}{2} [2\pi]$\\
$P \in O(2) \Leftrightarrow P = \begin{pmatrix}
    cos \alpha & -sin \alpha\\
    sin \alpha & cos \alpha\\
\end{pmatrix}_{det=1}$ ou $\begin{pmatrix}
    cos \alpha & sin \alpha\\
    sin \alpha & -cos \alpha
\end{pmatrix}_{det = -1}$

\subsection{\underline{Classification en dimension 2}}

\paragraph{\underline{Théorème}}: Soit E euclidien de dim 2 et $f \in O(E)$, $\mathcal{B}$ une BOND
\begin{enumerate}
    \item $f \in SO(E) \Leftrightarrow \exists \theta \in \mathbb{R}$ tq\\
$Mat_{\mathcal{B}} f = \begin{pmatrix}
    cos \theta & -sin \theta\\
    sin \theta & cos \theta\\
\end{pmatrix} = R_\theta$
Dans ce cas f est une "rotation d'angle $\theta$ et si $f \neq Id_E$ et $f \neq -Id_E$ f n'a pas de vap réelle
    \item $f \in O(E) \backslash SO(E) \Leftrightarrow \theta \in \mathbb{R}$ tq\\
    $Mat_{\mathcal{B}} f = \begin{pmatrix}
        cos \theta & sin \theta\\
        sin \theta & -cos \theta\\
\end{pmatrix}$\\
Donc dans ce cas f est une réflexion par rapport à la droite vectorielle $D = Ker(f - Id_E)$\\
De plus il existe une BOND $\mathcal{B}'$ dans laquelle $Mat_{\mathcal{B}'} f = \begin{pmatrix}
    1 & 0\\
    0 & -1\\
\end{pmatrix}$
\end{enumerate}

\begin{enumerate}
    \item $Mat_{\mathcal{B}} f = \begin{pmatrix}
    cos \theta & -sin \theta\\
    sin \theta & cos \theta\\
\end{pmatrix}$\\
$\chi_f(x) = \left| \begin{matrix}
        x - cos \theta & sin \theta\\
        -sin \theta & X - cos \theta\\
    \end{matrix}
\right|$\\
$\tab = X^2 -2X cos \theta + cos^2 \theta + sin^2 \theta$\\
$\tab = X^2 -2X cos \theta + 1$\\
$\tab = (X - e^{i\theta})(X - e^{-i \theta})$\\
Donc X est valeur propre réelle ssi $e^{i\theta} \in \mathbb{R}$
\end{enumerate}

$P = \begin{pmatrix}
    cos \theta & sin \theta\\
    sin \theta & -cos \theta\\
\end{pmatrix}$\\
det $(X I_2 - P) = \left|
    \begin{matrix}
        X - cos \theta & \sin \theta\\
        -sin \theta & X + cos \theta\\
    \end{matrix}
\right|$ = $X^2 -1$\\
$E = Ker(f - Id_E) \oplus Ker (f + Id_E)$\\
De plus il existe une BOND $\mathcal{B}'$ dans laquelle $Mat_{\mathcal{B}'} f = \begin{pmatrix}
    1 & 0\\
    0 & -1\\
\end{pmatrix}$

\subsection{\underline{Compléments}}

\begin{itemize}
    \item SO(2) = $\{\alpha \in \mathbb{R} | R_{\alpha}\}$
    \item $R_{\alpha} R_{\beta} = R_{\alpha + \beta} = R_{\beta} R_{\alpha}$
    \item $R_0 = I_2$
    \item Si $f \in SO(E)$ $\mathcal{B}$, $\mathcal{B}'$ BOND\\
Alors $R_{\theta} = Mat_{\mathcal{B}} f = Mat_{\mathcal{B}'} f = R_{\theta}'$ (*)
\end{itemize}
(*) La matrice de passage d'une BOND a une autre est dans SO(E) donc\\
$P^{-1} R_{\theta} P = R_{\theta'}$\\
$R_{-\alpha} R_{\theta} R_{\alpha} = R_{\theta'}$\\
$R_{-\alpha} R_{\alpha} R_{\theta} = R_{\theta'}$\\
$R_{\theta} = R_{\theta'}$\\
\\
$f \in O(E) \backslash SO(E)$\\
P = $\begin{pmatrix}
    cos \theta & sin \theta\\
    sin \theta & -cos \theta\\
\end{pmatrix}$\\
D = $Ker(f - Id_E)$\\
Soit $u_{\alpha} = cos (\alpha) \vec{i} + sin (\alpha) \vec{j}$\\
$(P - I_2) \cdot \begin{pmatrix}
    cos \alpha\\
    sin \alpha\\
\end{pmatrix}$\\
$P \begin{pmatrix}
cos (\alpha)\\
sin (\alpha)\\
\end{pmatrix}$
%Faire la suite

\end{document}
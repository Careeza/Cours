\documentclass{article}
\usepackage[utf8]{inputenc}
\usepackage{amsfonts}
\usepackage{amsmath}
\usepackage{graphicx}
\usepackage[a4paper, total={6in, 8in}]{geometry}
\usepackage{setspace}

\newcommand\tab[1][1cm]{\hspace*{#1}}
\newcommand{\scalaire}[2]{\left< #1 | #2\right>}
\onehalfspacing
\author{Frederic Becerril}

\begin{document}

\part*{Chapitre 4: Isométries vectorielles}

$(E, <|>)$ espace euclidien de dimension n, $||.||$ norme associée

\section{Isométries (vectorielle)}

\subsection{Définition} Un endomorphisme de E, $f \in \mathcal{L}(E)$ est une isométrie vectorielle ssi\\
\begin{align*}
    \forall x \in E, \; ||f(x)|| = ||x||
\end{align*}

\subsection{\underline{Remarque}} $f$ est un isomorphisme $(\in GL(E))$
\begin{align*}
    f : E \rightarrow E
\end{align*}
$x \in ker f \tab \underset{0}{\underbrace{||f(x)||}} = ||x||$\\
$\Leftrightarrow x = 0_E$

\subsection{\underline{Théorème:}} Soit $f \in \mathcal{L}(E)$, Les 4 propositions suivantes sont équivalentes:
\begin{enumerate}
    \item $f$ est une isométrie
    \item $f$ conserve le produit scalaire:\\
    $\forall(x, y) \in E^2 <f(x)|f(y)> = <x|y>$
    \item $f$ transforme une BON en une BON
    \item La matrice représentative de $f$ dans une BON est orthogonal
\end{enumerate}

\newpage

\subsection{\underline{Preuve:}}
On va montrer $(1) \Rightarrow (2) \Rightarrow (3) \Rightarrow (4) \Rightarrow (1)$
\begin{itemize}
    \item $(1) \Rightarrow (2)$\\
    $\forall (x, y) \in E^2 \tab <x|y> = \frac{1}{2}(||x+y||^2 - ||x||^2 - ||y||^2)$\\
    $\forall (x, y) \in E^2 \tab <f(x), f(y)> \; = \frac{1}{2}(||f(x) + f(y)||^2 - ||f(x)||^2 - ||f(y)||^2)$\\
    $\tab[5.3cm] =\frac{1}{2}(||f(x+y)||^2 - ||f(x)||^2 - ||f(y)||^2)$ (*)\\
    $\tab[5.3cm] =\frac{1}{2}(||x+y||^2 - ||x||^2 - ||y||^2)$ (**)\\
    $\tab[5.3cm] = \; <x|y>$
    \item $(2) \Rightarrow (3)$\\
    $\{u_1, \dots, u_n\}$ BON de E\\
    On pose $\forall i \in \{1, \dots, n\}, \; v_i = f(u_i)$\\
    $||v_i||^2 = <v_i|v_i> = <f(u_i)|f(u_i)> = <u_i|u_i> = 1$\\
    Si $i \neq j$\\
    $<v_i | v_j> = <f(u_i)|f(u_j)> = <u_i|u_j> = 0$\\
    $\{v_1, \dots, v_n\}$ famille orthonormée de $n = dim E$ vecteurs
    \item $(3) \Rightarrow (4)$\\
    $\mathcal{B} = \{u_1, \dots, u_n\}$ une BON\\
    $P = Mat_{\mathcal{B}}f = \begin{pmatrix}
        f(u_1) & \dots & f(u_n)\\
        \vdots & & \vdots\\
    \end{pmatrix}$\\
    matrice de passage de $\{u_1, \dots, u_n\}$ à $v_1, \dots, v_n$ BON donc la matrice est orthogonale
    \item $(4) \Rightarrow (1)$\\
    Si $P = Mat_{\mathcal{B}}f \in O(n)$\\
    Si $\mathcal{B} = \{u_1, \dots, u_n\}$ est une BON\\
    $x \in E \rightarrow X = \begin{pmatrix}
        x_1\\
        \vdots\\
        x_n\\
    \end{pmatrix}$ tq $x = x_1 u_1 + \dots + x_n u_n$\\
    Si $y = f(x) \rightarrow Y = PX$\\
    Si BON $||x||^2 = <x|x> = \sum_{i = 1}^n x_i^2 = X^T . X$\\
    $||f(x)||^2 = ||y||^2 = Y^T \cdot Y = (PX)^T \cdot (PX) = X^T \cdot P^T \cdot P \cdot X = X^T \cdot X = ||x||^2$
\end{itemize}
(*) Linéarité de f\\
(**) Isométrie de f

\newpage

\subsection{\underline{Théorème:}} 
\begin{itemize}
    \item  Dans E euclidien, l'ensemble des isométries vectorielles de E est noté O(E) et c'est un sous-groupe de GL(E)
    \item Si B est une BON de E\\
    $O(E) \rightarrow O(n)$\\
    $f \longmapsto Mat_B(f)$
    Est un isomorphisme de groupe
\end{itemize}

\subsection{\underline{Propriétés supplémentaire:}} 

\begin{enumerate}
    \item Si $f \in O(E)$ $det f = \pm 1$, $SO(E) = \{f \in O(E) \mbox{ tq } det f = 1\}$ sous groupe de $O(E)$\\
    \tab[5cm] $\rightarrow$ isométrie positives ou directe
    \item Si $det f =-1$\\
    \tab[1cm] $\rightarrow$ isométrie négative ou indirecte
    \item Si $f \in O(E)$ et $F \subset E$ est un sev stable par f\\
    $F^\perp$ est stable par f.\\
    $f_{|F} F \rightarrow F$ bijective $(f: E \rightarrow E)$ bijective
    Soit  $z \in F^\perp$ on veut montrer que $f(z) \in F^\perp$\\
    $\forall x \in F$, $\exists y \in F$ tq $f(y) = x$\\
    $\scalaire{f(z)}{x} = \scalaire{f(z)}{f(y)} = \scalaire{z}{y} = 0$, $f(z) \in F^\perp$
    \item Si $f$ a des valeurs propres réelles. Elles peuvent valoir que 1 où $-1$\\
    Si $\lambda \in \mathbb{R}$ et $x \neq 0$ tq $f(x) = \lambda x$\\
    $\underset{\|}{||f(x)||} = ||x||$\\
    $\tab[1.5mm]||\lambda x|| = |\lambda| \; ||x|| \Rightarrow |\lambda| = 1$
    \item $\ker(f - Id_E)$ et $\ker(f + Id_E)$ sont orthogonaux
\end{enumerate}

\subsection{\underline{Les symétries orthogonales:}}

$F$ sev de $E$; $E = F \oplus F^\perp$; $P_F$ et $P_{F^\perp}$\\
$S_F$ = $P_F - P_{F^\perp} = 2P_F - Id_E = Id_E - 2P_{F^\perp}$

\paragraph{\underline{Propriétés:}} $S_F$ est une isométrie vectorielle\\
$||S_F(x)||^2 = ||P_F(x) - P_{F^\perp}(x)||^2 = ||P_F(x)||^2 + ||P_{F^\perp}(x)||^2$\\
$\tab[1.6cm] = ||P_F(x) + P_{F^\perp}(x)||^2 = ||x||^2$\\
Si $S$ est une symétries $\perp$. C'est la symétries $\perp$ par rapport à $F = \ker\{S - Id_E\}$ et $F^\perp = \ker\{S + Id_E\}$\\
Si $f \in O(E)$ et $\ker\{f - Id_E\} \oplus \ker\{f + Id_E\} = E$ $f$ est la symétrie $\perp$ par rapport à $\ker \{f - Id_E\}$\\
Si $f$ est une symétrie $\perp$ il existse une $BON$ $\mathcal{B}$ tq\\
$Mat_{\mathcal{B}}f = \begin{pmatrix}
    1 & 0 & \dots & \dots & \dots & 0\\
    0 & \ddots & & & & 0\\
    \vdots & & 1 & & & \vdots\\
    \vdots & & & -1 & & \vdots\\
    \vdots & & & & \ddots & \vdots\\
    0 & \dots & \dots & \dots & 0 & -1\\
\end{pmatrix}$ \vspace{3mm}\\
Voc: Une symétrie par rapport à $F$ hyperplan s'appelle réflexion   



\end{document}
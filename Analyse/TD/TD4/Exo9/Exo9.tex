\documentclass{article}
\usepackage[utf8]{inputenc}
\usepackage{amsfonts}
\usepackage{amsmath}
\usepackage{graphicx}
\usepackage[a4paper, total={6in, 8in}]{geometry}
\usepackage{setspace}

\everymath{\displaystyle}

\newcommand\tab[1][1cm]{\hspace*{#1}}
\doublespacing
\author{Frederic Becerril}

\newcommand{\mylim}[2]{\underset{#1 \rightarrow #2}{\longrightarrow}}
\newcommand{\mysim}[2]{\underset{#1 \rightarrow #2}{\sim}}
\newcommand{\mysupp}[1]{\underset{x \in #1}{\sup}}
% \newcommand{\citer}[2]{\og #2 \fg{} (#1)}

\begin{document}

\part*{Exerice 9}

Soit $R > 0$\\
$P_n : x \mapsto \sum_{k = 0}^{n} \frac{x^k}{k!}$\\
\\
On sait que $\underset{n \rightarrow \infty}{\lim} P_n$ est la série entière de $e^x$ qui a pour Rayon de convergence $R = +\infty$\\
Donc $P_n$ converge uniformément sur $\mathbb{R}$ vers la fonction $e^x$\\
On a donc que $\exists n_R \in \mathbb{N}$ tq pour tout les entiers $n \geq n_R$, les fonctions $P_n - e$ soit bornée sur [-R, R]\\
Càd: soit $\epsilon = \exp\left(\frac{-R}{2}\right) > 0$ on a que $\exists n_R \in \mathbb{N}, \forall n \geq n_R$ on a $||P_n - e||_{\infty, [-R, R]} < \epsilon$\\
Or $e$ est strictement croissant sur $\mathbb{R}$, donc $\forall x \in [-R, R]$, $e^x \geq e^{-R}$\\
$\epsilon > ||P_n - e||_{\infty} \geq ||P_n - e^{-R}||_{\infty}$\\
$\Rightarrow \forall x \in [-R, R]$ on a que $|P_n(x) - e^{-R}| < \epsilon$\\
$\Rightarrow$ $e^{-R} - \epsilon < P_n(x) < e^{-R} + \epsilon$\\
$\Rightarrow$ $e^{-R} - e^{-R/2} < P_n(x) < e^{-R} + e^{-R/2}$\\
Or $e^{-R} - e^{-R/2} > 0$, donc:\\
$\Rightarrow$ $0 < P_n(x) < e^{-R} + e^{-R/2}$\\
Donc $P_n(x)$ n'admet pas de racine réelle sur $[-R, R]$ $\forall n \geq n_R$ 
\end{document}
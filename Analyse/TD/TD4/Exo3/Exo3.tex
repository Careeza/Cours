\documentclass{article}
\usepackage[utf8]{inputenc}
\usepackage{amsfonts}
\usepackage{amsmath}
\usepackage{graphicx}
\usepackage[a4paper, total={6in, 8in}]{geometry}
\usepackage{setspace}

\everymath{\displaystyle}

\newcommand\tab[1][1cm]{\hspace*{#1}}
\doublespacing
\author{Frederic Becerril}

\newcommand{\mylim}[2]{\underset{#1 \rightarrow #2}{\longrightarrow}}
\newcommand{\mysim}[2]{\underset{#1 \rightarrow #2}{\sim}}
% \newcommand{\citer}[2]{\og #2 \fg{} (#1)}

\begin{document}

\part*{Exerice 3}

Soit $R_1$ le rayon de convergence de $\sum a_n x_n$\\
$x^n a_n$ est borné si $x \in ]-R_1, R_1[$ et n'est pas borné si $x \in \mathbb{R} \backslash [-R_1, R_1]$\\
Soit $x \in \mathbb{R}_+$ et $x' = x^2$\\
Si $x' < \sqrt{R_1} \Rightarrow x < R_1$\\
Donc $(x')^n a_n$ est borné\\
Si $x' > \sqrt{R_1} \Rightarrow x > R_1$, donc $x \in \mathbb{R} \backslash [-R_1, R_1]$\\
Donc $(x')^n a_n$ n'est pas borné\\
Alors $sup \{x'\in \mathbb{R}_+ \;|\; a_n x'_n \mbox{ borné }\} = sup \{x\in \mathbb{R}_+ \;|\; a_n (x^2)^n \mbox{ borné }\} = \sqrt{R_1} = R_2$\\
\end{document}
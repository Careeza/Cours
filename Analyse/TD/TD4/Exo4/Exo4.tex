\documentclass{article}
\usepackage[utf8]{inputenc}
\usepackage{amsfonts}
\usepackage{amsmath}
\usepackage{graphicx}
\usepackage[a4paper, total={6in, 8in}]{geometry}
\usepackage{setspace}

\everymath{\displaystyle}

\newcommand\tab[1][1cm]{\hspace*{#1}}
\doublespacing
\author{Frederic Becerril}

\newcommand{\mylim}[2]{\underset{#1 \rightarrow #2}{\longrightarrow}}
\newcommand{\mysim}[2]{\underset{#1 \rightarrow #2}{\sim}}
% \newcommand{\citer}[2]{\og #2 \fg{} (#1)}

\begin{document}

\part*{Exerice 4}

$\bullet$ $\sum_{n\geq 1} \frac{1}{n3^n}x^n$\\
$a_n = \frac{1}{n3^n}$\\
Étudions $\frac{a_{n+1}}{a_n} = \frac{n3^n}{(n+1)3^{n+1}} = \frac{n3^n}{(n+1)3^n * 3} = \frac{n}{3(n+1)} \mylim{n}{\infty} \frac{1}{3}$\\
Le critère de d'Alembert pour les séries entière nous dis que le rayon de convergence est R = $\frac{1}{\frac{1}{3}} = 3$\\
Il nous suffit donc d'étudier $\sum_{n\geq 1} \frac{1}{n3^n}x^n$ en $x = -3$ et $x = 3$
\paragraph{\underline{Cas x = 3:}} $\sum_{n\geq 1} \frac{1}{n3^n}3^n$
$=\sum_{n\geq 1} \frac{1}{n}$\\
Cette somme diverge, donc $3 \notin O$
\paragraph{\underline{Cas x = -3:}} $\sum_{n\geq 1} \frac{1}{n3^n}3^n$
$=\sum_{n\geq 1} \frac{(-1)^n}{n}$\\
Cette somme converge, donc $3 \in O$\\
Donc O = $[-3, 3[$\\
\\
Soit $S = \sum_{n\geq 1} \frac{1}{n3^n}x^n$, S est de classe $C^\infty$ sur $]-3, 3[$\\
Et on a que $S'(x) = \sum_{n\geq 1} n\frac{1}{n3^n}x^{n - 1} = \sum_{n\geq 1} \frac{1}{3^n}x^{n - 1} = \sum_{n\geq 0} \frac{1}{3^{n+1}}x^n = \frac{1}{3} \sum_{n\geq 0} \frac{1}{3^n}x^n = \frac{1}{3} \sum_{n\geq 0} (\frac{x}{3})^n$\\
On reconnait une suite géométrique de raison $\frac{x}{3}$\\
Or on a que $\forall x \in ]-3, 3[$, $|\frac{x}{3}| < 1$\\
Donc $S'(x) = \frac{1}{3} \times \frac{1}{1-q}$, avec $q = \frac{x}{3}$\\
Donc $S'(x) = \frac{1}{3} \times \frac{1}{1 - \frac{x}{3}} = \frac{1}{3 - x}$\\
Or $-ln(3 - x)' = \frac{1}{3-x}$\\
Donc $S(x) = -ln(3 - x) + c$, avec $c \in \mathbb{R}$\\
On sait que $S(0) = 0$\\
Donc on cherche $c \in \mathbb{R}$ tq $-ln(3 - 0) + c = 0 \Leftrightarrow c = ln(3)$\\
Donc $S(x) = -ln(3 - x) + ln(3) = -(ln(3 - x) - ln(3)) = -ln(\frac{3 - x}{3})$\\

\noindent $\bullet$ $\sum_{n\geq 0} \frac{n + 2}{n + 1}x^n$
$= \sum_{n\geq 0} \frac{n + 1 + 1}{n + 1}x^n$
$= \sum_{n\geq 0} (\frac{1}{n + 1} + 1)x^n $
$= \sum_{n\geq 0} \frac{1}{n + 1}x^n  +\sum_{n\geq 0} x^n$\\
$\tab[2.35cm] = \sum_{n\geq 0} \frac{1}{n + 1}x^n  + \frac{1}{1-x}$


\end{document}
\documentclass{article}
\usepackage[utf8]{inputenc}
\usepackage{amsfonts}
\usepackage{amsmath}
\usepackage{graphicx}
\usepackage[a4paper, total={6in, 8in}]{geometry}
\usepackage{setspace}

\everymath{\displaystyle}

\newcommand\tab[1][1cm]{\hspace*{#1}}
\doublespacing
\author{Frederic Becerril}

\newcommand{\mylim}[2]{\underset{#1 \rightarrow #2}{\longrightarrow}}
\newcommand{\mysim}[2]{\underset{#1 \rightarrow #2}{\sim}}
\newcommand{\mysupp}[1]{\underset{x \in #1}{\sup}}
% \newcommand{\citer}[2]{\og #2 \fg{} (#1)}

\begin{document}

\part*{Exerice 8}

$\frac{x}{(x-1)(x-2)} = \frac{a}{x - 1} + \frac{b}{x -  2}$ \vspace{3mm}\\
$\frac{a(x- 2)+b(x - 1)}{(x-1)(x-2)} = \frac{ax -2a +bx -b}{(x-1)(x-2)} = \frac{x(a + b) -2a -b}{(x-1)(x-2)}$ \vspace{3mm}\\
$\left\{
    \begin{array}{ll}
        a + b = 1\\
        -2a - b = 0\\
    \end{array}
\right. \Leftrightarrow \left\{
    \begin{array}{ll}
        a = 1 - b\\
        -2(1 - b) - b = 0\\
    \end{array}
\right. \Leftrightarrow \left\{
    \begin{array}{ll}
        a = -1\\
        b = 2\\
    \end{array}
\right.$\\
Donc $\frac{x}{(x-1)(x-2)} = \frac{-1}{x - 1} + \frac{2}{x -  2}$\\
$ \frac{-1}{x - 1} = \frac{1}{1 - x} = \sum_{n \geq 0} x^n$\\
$ \frac{2}{x -  2} = -\frac{1}{1 - \frac{x}{2}} = -\sum_{n \geq 0} \left(\frac{x}{2}\right)^n$\\
$\frac{x}{(x-1)(x-2)} = \sum_{n \geq 0} x^n - \sum_{n \geq 0} \left(\frac{x}{2}\right)^n = \sum_{n \geq 0} x^n - \left(\frac{x}{2}\right)^n$\\
On a que le rayon de convergence de $\sum_{n \geq 0} \left(\frac{x}{2}\right)^n$ est $2$\\
On a que le rayon de convergence de $\sum_{n \geq 0} x^n$ est $1$\\
Donc le rayon de convergence est $\min(1, 2) = 1$\\
$\frac{1}{1 + x + x^2}$\\
$1 + x + x^2 = \left(x + \frac{1}{2}\right)^2 + \frac{3}{4}$\\
$\tab[1.8cm] = \left(x + \frac{1}{2}\right) - \left(\frac{i\sqrt{3}}{2}\right)^2$\\
$\tab[1.8cm] = \left(x + \frac{1}{2} - \frac{i\sqrt{3}}{2}\right) \left(x + \frac{1}{2} + \frac{i\sqrt{3}}{2}\right)$\\
$\tab[1.8cm] = \left(x + \frac{1- i\sqrt{3}}{2}\right) \left(x + \frac{1 + i\sqrt{3}}{2}\right)$\\
$\frac{1}{1 + x + x^2} = \frac{1}{\left(x + \frac{1- i\sqrt{3}}{2}\right) \left(x + \frac{1 + i\sqrt{3}}{2}\right)}$\\
$\frac{1}{1 + x + x^2} = \frac{a}{\left(x + \frac{1- i\sqrt{3}}{2}\right)} + \frac{b}{\left(x + \frac{1 + i\sqrt{3}}{2}\right)}$\\
$\frac{1}{1 + x + x^2} = \frac{\left(x + \frac{1 + i\sqrt{3}}{2}\right)a + \left(x + \frac{1- i\sqrt{3}}{2}\right)b}{\left(x + \frac{1- i\sqrt{3}}{2}\right)\left(x + \frac{1 + i\sqrt{3}}{2}\right)}$\\
$\frac{1}{1 + x + x^2} = \frac{\left(x + \frac{1 + i\sqrt{3}}{2}\right)a + \left(x + \frac{1- i\sqrt{3}}{2}\right)b}{1 + x + x^2}$\\
$\frac{1}{1 + x + x^2} = \frac{ax + a\frac{1 + i\sqrt{3}}{2} + bx + b\frac{1- i\sqrt{3}}{2}}{1 + x + x^2}$\\
$\left\{
    \begin{array}{ll}
        a + b = 0\\
        a\frac{1 + i\sqrt{3}}{2} + b\frac{1 - i\sqrt{3}}{2} = 1\\
    \end{array}
\right. \Leftrightarrow \left\{
    \begin{array}{ll}
        a = - b\\
        -b\frac{1 + i\sqrt{3}}{2} + b\frac{1 - i\sqrt{3}}{2} = 1\\
    \end{array}
\right.$\\
$\Leftrightarrow \left\{
    \begin{array}{ll}
        a = - b\\
        \frac{-b - ib\sqrt{3}}{2} + \frac{b - ib\sqrt{3}}{2} = 1\\
    \end{array}
\right.$\\
$\Leftrightarrow \left\{
    \begin{array}{ll}
        a = - b\\
        \frac{-2ib\sqrt{3}}{2} = 1\\
    \end{array}
\right.$
$\Leftrightarrow \left\{
    \begin{array}{ll}
        a = - b\\
        ib\sqrt{3} = -1\\
    \end{array}
\right.$\\
$\Leftrightarrow \left\{
    \begin{array}{ll}
        a = - b\\
        b = \frac{i}{\sqrt{3}}\\
    \end{array}
\right.$
$\Leftrightarrow \left\{
    \begin{array}{ll}
        a = \frac{-i\sqrt{3}}{3}\\
        b = \frac{i\sqrt{3}}{3}\\
    \end{array}
\right.$ \vspace{3mm}\\
$\frac{1}{1 + x + x^2} = -\frac{\frac{i\sqrt{3}}{3}}{\left(x + \frac{1- i\sqrt{3}}{2}\right)} + \frac{\frac{i\sqrt{3}}{3}}{\left(x + \frac{1+i\sqrt{3}}{2}\right)}$

\end{document}
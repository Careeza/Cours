\documentclass{article}
\usepackage[utf8]{inputenc}
\usepackage{amsfonts}
\usepackage{amsmath}
\usepackage{graphicx}
\usepackage[a4paper, total={6in, 8in}]{geometry}
\usepackage{setspace}

\everymath{\displaystyle}

\newcommand\tab[1][1cm]{\hspace*{#1}}
\doublespacing
\author{Frederic Becerril}

\newcommand{\mylim}[2]{\underset{#1 \rightarrow #2}{\longrightarrow}}
\newcommand{\mysim}[2]{\underset{#1 \rightarrow #2}{\sim}}
% \newcommand{\citer}[2]{\og #2 \fg{} (#1)}

\begin{document}

\part*{Exerice 1}

$\bullet$ $\sum_{n \geq 2} \frac{x^n}{ln(n)}$\\
$a_n = \frac{1}{ln(n)}$\\
Calculons $\frac{a_{n+1}}{a_n} = \frac{ln(n+1)}{ln(n)}$\\
Or $ln(n + 1) \underset{n \rightarrow \infty}{\sim} ln(n)$\\
Donc $\frac{ln(n+1)}{ln(n)} \underset{n \rightarrow \infty}{\rightarrow} 1$\\
Le critère de d'Alembert pour les séries entière nous dis que le rayon de convergence est R = $\frac{1}{1} = 1$\\
$\bullet$ $\sum_{n \geq 1} \frac{(-1)^n}{n^n} x^{2n + 1}$\\
$\left\{
    \begin{array}{ll}
        b_n = \frac{(-1)^n}{n^n} \mbox{ si } n = 2k + 1 \mbox{ avec } k \geq 1\\
        b_n = 0 \mbox{ si sinon}\\
    \end{array}
\right.$\\
Calculons $\sqrt[n]{|b_n|} = \sqrt[n]{\frac{1}{n^n}} = \frac{1}{n} \underset{n \rightarrow \infty}{\longrightarrow} 0 \tab \mbox{ou } \sqrt[n]{0} = 0\underset{n \rightarrow \infty}{\longrightarrow} 0$\\
Le critère de Cauchy nous dit que le rayon de convergence R est égale à $\frac{1}{0}$\\
Qui par convention = $\infty$\\
$\bullet \sum_{n \geq 1} (2^n - n)x^n$\\
$c_n = 2^n - n$\\
Calculons $\frac{c_{n+1}}{c_n} = \frac{2^{n + 1} - (n + 1)}{2^n - n} = \frac{2 * 2^n -n + 1}{2^n - n} = 2\frac{2^n}{2^n - n} - \frac{n - 1}{2^n - n}$\\
Or $2^n - n \underset{n \rightarrow \infty}{\sim} 2^n$ par croissance comparé\\
Donc $\frac{c_{n+1}}{c_n} \underset{n \rightarrow \infty}{\longrightarrow} 2$\\
Le critère de d'Alembert pour les séries entière nous dis que le rayon de convergence est R = $\frac{1}{2}$\\
$\bullet$ $\sum_{n \geq 0} n^{\sqrt{n}} x^n$\\
$d_n = n^{\sqrt{n}}$\\
Calculons $\sqrt[n]{n^{\sqrt{n}}} = (n^{\sqrt{n}})^{\frac{1}{n}} = n^{\frac{\sqrt{n}}{n}} = n^{\frac{1}{\sqrt{n}}} = exp(\frac{ln(n)}{\sqrt{n}})$\\
Or $\frac{ln(n)}{\sqrt{n}} \underset{n \rightarrow \infty}{\longrightarrow} = 0$, donc $exp(\frac{ln(n)}{\sqrt{n}}) \underset{n \rightarrow \infty}{\longrightarrow} 1$\\ 
Le critère de Cauchy nous dit que le rayon de convergence R est égale à $\frac{1}{1} = 1$\\
\newpage

\noindent $\bullet$ $\sum_{n \geq 0} 3^n x^{n!}$\\
$\left\{
    \begin{array}{ll}
        f_n = 3^k \mbox{ si } \exists k \in \mathbb{N} \mbox{ tq } n = k!\\
        f_n = 0 \mbox{ si sinon}\\
    \end{array}
\right.$\\
Donc $a_k x_k$ est borné ssi $3^k x^{k!} = (3x)^k * x^{(k-1)!}$ est borné\\
\noindent Étudions $sup \{x\in \mathbb{R}_+ \;|\; a_n x_n \mbox{ borné }\}$\\
Si $x = 1$ on $a_k x_k = 3^k \mylim{k}{\infty} \infty$ et donc $a_k x_k$ n'est pas borné\\
Si $x > 1$ on $(3x)^k * x^{(k-1)!} \mylim{k}{\infty} \infty$, donc $a_k x_k$ n'est pas borné\\
Si $x < 1$ donc $x^{(k-1)!} \mylim{k}{\infty} 0$, par croissance comparé on a $a_k x_k \mylim{k}{\infty} 0$ donc est borné\\
$sup \{x\in \mathbb{R}_+ \;|\; a_n x_n \mbox{ borné }\} = 1$\\
$R = 1$\\

\noindent $\bullet$ $\sum_{n \geq 0} x^{n^2}$\\
$\left\{
    \begin{array}{ll}
        f_n = 1 \mbox{ si } \exists k \in \mathbb{N} \mbox{ tq } n = k^2\\
        f_n = 0 \mbox{ si sinon}\\
    \end{array}
\right.$\\
Donc $a_n x_n$ est borné ssi $x^n$ est borné\\
\noindent Étudions $sup \{x\in \mathbb{R}_+ \;|\; a_n x_n \mbox{ borné }\}$\\
Si $x = 1$ on a $a_n x_n = 1$ et donc $a_n x_n$ est borné\\
Si $x > 1$ on a $x^n \mylim{n}{\infty} \infty$, donc $a_n x_n$ n'est pas borné\\
Si $x < 1$ on a $x^n \mylim{n}{\infty} 0$, donc $a_n x_n$ est borné\\
$sup \{x\in \mathbb{R}_+ \;|\; a_n x_n \mbox{ borné }\} = 1$\\
$R = 1$

\newpage
\noindent $\bullet$ $\sum_{n \geq 1} (tan(\frac{1}{n}) - sin(\frac{1}{n})) x^n$\\
$g_n = tan(\frac{1}{n}) - sin(\frac{1}{n})$\\
$\sqrt[n]{g_n} = (tan(\frac{1}{n}) - sin(\frac{1}{n}))^\frac{1}{n}$\\
$= (\frac{1}{n} + \frac{1}{3n^3} + o(\frac{1}{n^3}) - (\frac{1}{n} - \frac{1}{6n^3} + o(\frac{1}{n^3})))^\frac{1}{n} = \sqrt[n]{\frac{1}{2n^3} + o(\frac{1}{n^3})}$\\
$\mysim{n}{\infty} \sqrt[n]{\frac{1}{2n^3}} = \frac{1}{2n^{3/n}} = \frac{1}{exp(\frac{3ln(2n)}{n})}$\\
or $\frac{3ln(2n)}{n} \mylim{n}{\infty} 0$ croissance comparé\\
donc $exp(\frac{3ln(2n)}{n}) \mylim{n}{\infty} 1$ et $\frac{1}{exp(\frac{3ln(2n)}{n})} \mylim{n}{\infty} 1$\\
Le critère de Cauchy nous dit que le rayon de convergence R est égale à $\frac{1}{1} = 1$\\
$\bullet$ $\sum_{n \geq 1} a_n x_n$\\
$a_n = \left\{
    \begin{array}{ll}
        n 2^n \tab \mbox{si n est impair}\\
        \frac{1}{n} \tab \mbox{si n est pair}\\
    \end{array}
\right.$\\
Calculons $\sqrt[n]{a_n}$\\
\underline{Cas 1}: n pair\\
$\sqrt[n]{a_n} = \frac{1}{\sqrt[n]{n}}$\\
Or $\sqrt[n]{n} = exp(\frac{ln(n)}{n}) \mylim{n}{\infty} 1$\\
Donc $\sqrt[n]{a_n} \mylim{n}{\infty} 1$\\
\underline{Cas 2}: n impair\\
$\sqrt[n]{a_n} = \sqrt[n]{2^n n} \mylim{n}{\infty} 2$\\
On a que $2 \neq 1$, donc cette méthode ne marche pas, autre méthode page suivant\\

\newpage

\noindent Étudions $sup \{x\in \mathbb{R}_+ \;|\; a_n x_n \mbox{ borné }\}$\\
\underline{Cas 1}: n pair\\
$a_n x_n = n 2^n x^n = n (2x)^n$\\
Si $x = \frac{1}{2}$ on $a_n = n \mylim{n}{\infty} \infty$ et donc $a_n x_n$ n'est pas borné\\
Si $x > \frac{1}{2}$ on $2x > 1$ donc $(2x)^n \mylim{n}{\infty} \infty$ alors $a_n x_n \mylim{n}{\infty} \infty$ et donc $a_n x_n$ n'est pas borné\\
Si $x < \frac{1}{2}$ on $2x < 1$ donc $(2x)^n \mylim{n}{\infty} 0$ et par croissance comparé $n (2x)^n \mylim{n}{\infty} 0$\\
Alors $a_n x_n \mylim{n}{\infty} 0$ donc $a_n x_n$ est borné\\
$sup \{x\in \mathbb{R}_+\mbox{, n pair } \;|\; a_n x_n \mbox{ borné }\} = \frac{1}{2}$\\
\underline{Cas 2}: n impair\\
$a_n x_n = \frac{1}{n} x^n$\\
Si $x = 1$ on $a_n = n \mylim{n}{\infty} \infty$ et donc $a_n x_n$ n'est pas borné\\
Si $x > 1$ on $2x > 1$ donc $(x)^n \mylim{n}{\infty} \infty$ alors $a_n x_n \mylim{n}{\infty} \infty$ et donc $a_n x_n$ n'est pas borné\\
Si $x < 1$ donc $(x)^n \mylim{n}{\infty} 0$ et par croissance comparé $n (x)^n \mylim{n}{\infty} 0$\\
Alors $a_n x_n \mylim{n}{\infty} 0$ donc $a_n x_n$ est borné\\
$sup \{x\in \mathbb{R}_+ \mbox{, n impair } \;|\; a_n x_n \mbox{ borné }\} = 1$\\
Donc dans le cas généarl on a:\\
$sup \{x\in \mathbb{R}_+ \;|\; a_n x_n \mbox{ borné }\}$\\
$= min(sup \{x\in \mathbb{R}_+ \mbox{, n impair } \;|\; a_n x_n \mbox{ borné }\}, sup \{x\in \mathbb{R}_+ \mbox{, n pair } \;|\; a_n x_n \mbox{ borné }\}) = \frac{1}{2}$\\
Donc $R = \frac{1}{2}$

\end{document}
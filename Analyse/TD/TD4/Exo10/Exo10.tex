\documentclass{article}
\usepackage[utf8]{inputenc}
\usepackage{amsfonts}
\usepackage{amsmath}
\usepackage{graphicx}
\usepackage[a4paper, total={6in, 8in}]{geometry}
\usepackage{setspace}

\everymath{\displaystyle}

\newcommand\tab[1][1cm]{\hspace*{#1}}
\doublespacing
\author{Frederic Becerril}

\newcommand{\mylim}[2]{\underset{#1 \rightarrow #2}{\longrightarrow}}
\newcommand{\mysim}[2]{\underset{#1 \rightarrow #2}{\sim}}
\newcommand{\mysupp}[1]{\underset{x \in #1}{\sup}}
% \newcommand{\citer}[2]{\og #2 \fg{} (#1)}

\begin{document}

\part*{Exerice 10}

$\sum_{n \geq 0} \frac{x^n}{(2n)!}$\\
$a_n = \frac{1}{(2n)!}$\\
$\frac{a_{n + 1}}{a_n} = \frac{n!}{(n + 1)!} = \frac{1}{n + 1} \mylim{n}{\infty} 0 = l$\\
Donc d'après d'Alembert on a que, le rayon de convergence de cette série entière est:\\
$R = \frac{1}{l} = \frac{1}{0} = +\infty$ par convention\\
$S(x) = \sum_{n \geq 0} \frac{x^n}{(2n)!}$\\
On fait un changement de variable $x' = \sqrt{x}$\\
$S(x') = \sum_{n \geq 0} \frac{x'^{2n}}{(2n)!} = cosh(x')$\\
Donc  $S(x) = cosh(\sqrt{x})$\\
\end{document}
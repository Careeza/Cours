\documentclass{article}
\usepackage[utf8]{inputenc}
\usepackage{amsfonts}
\usepackage{amsmath}
\usepackage{graphicx}
\usepackage[a4paper, total={6in, 8in}]{geometry}
\usepackage{setspace}

\everymath{\displaystyle}

\newcommand\tab[1][1cm]{\hspace*{#1}}
\doublespacing
\author{Frederic Becerril}

\newcommand{\mylim}[2]{\underset{#1 \rightarrow #2}{\longrightarrow}}
\newcommand{\mysim}[2]{\underset{#1 \rightarrow #2}{\sim}}
\newcommand{\mysupp}[1]{\underset{x \in #1}{\sup}}
% \newcommand{\citer}[2]{\og #2 \fg{} (#1)}

\begin{document}

\part*{Exerice 7}

$\begin{matrix}
    1) \;f(x) = \frac{1}{1 - x} & f(0) = 1\\
    f'(x) = \frac{1}{(1 - x)^2} & f'(0) = 1\\
    f''(x) = \frac{2}{(1 - x)^3} & f''(0) = 2\\
    f^{(3)}(x) = \frac{2 * 3}{(1 - x)^4} & f^{(3)}(0) = 6\\
    f^{(n)}(x) = \frac{n!}{(1 - x)^n} & f^{(n)}(0) = n!\\
\end{matrix}$\\
On a que $S(x) = \sum_{n \geq 0} a_n x^n$ avec $a_n = \frac{S^{(n)}(0)}{n!}$\\
Soit $f(x) = \frac{1}{1 -x} = \sum_{n \geq 0} \frac{f^{(n)}(0)}{n!} x^n = \sum_{n \geq 0} \frac{n!}{n!}x^n = \sum_{n \geq 0} x^n$\\
Étudions son rayon de convergence:\\
On a $a_n = 1$\\
Donc $\sqrt[n]{1} = 1$\\
Donc d'après Riemann le rayon de convergence $R = \frac{1}{1} = 1$\\
$f(1) = \sum_{n \geq 0} 1^n = \sum_{n \geq 0} 1$ diverge\\
$f(-1) = \sum_{n \geq 0} (-1)^n = \sum_{n \geq 0} 1$ diverge\\
Donc le domaine de convergence est $]-1, 1[$\\
2) $(\frac{1}{1 - x})' = \frac{1}{(1 - x)^2}$\\
Soit $x \in ]-1, 1[$\\
Donc $\frac{1}{(1 - x)^2} = \left(\sum_{n \geq 0} x^n\right)' = \sum_{n \geq 0} n x^{n - 1}$\\
On a donc que le rayon de convergence de cette série est le même que celui de $\sum_{n \geq 0} x^n$\\
$f(1) = \sum_{n \geq 0} n$ diverge\\
$f(-1) = \sum_{n \geq 0} (-1)^n n$ diverge, car $n$ ne tend pas vers 0\\
Donc le domaine de convergence est $]-1, 1[$\\
\newpage
\noindent 3) $ln(1+x)' = \frac{1}{1+x}$\\
Soit $x \in ]-1, 1[$\\
On fait un changement de variable $x' = -x$\\
$\frac{1}{1 - x'} = \sum_{n \geq 0} x'^n$\\
$\frac{1}{1 + x} = \sum_{n \geq 0} (-x)^n$\\
$\frac{1}{1 + x} = \sum_{n \geq 0} (-1)^n x^n$\\
On a que $ln(1 + x) - ln(1)$ = $\int_0^{x} \sum_{n \geq 0} (-1)^n t^n dt = \sum_{n \geq 0} \frac{(-1)^n}{n + 1} x^{n + 1}$\\
$ln(1 + x) = \sum_{n \geq 0} \frac{(-1)^n}{n + 1} x^{n + 1} = \sum_{n \geq 1} (-1)^{n - 1} \frac{x^n}{n}$\\
4) Le développement en série entière de $e^x$ est connu:\\
On a $e^x = \sum_{n \geq 0} \frac{x^n}{n!}$\\
Son rayon de convergence est $R = +\infty$\\
Et son domaine de convergence est $\mathbb{R}$\\
5) $\cos(x)$\\
$\cos(\theta) = \frac{e^{i \theta} + e^{-i \theta}}{2} = \frac{1}{2} \left(\sum_{n \geq 0} \frac{(i \theta)^n}{n!} + \sum_{n \geq 0} \frac{(-i \theta)^n}{n!}\right)$\\
$\tab[3.05cm] = \frac{1}{2} \left(\sum_{n \geq 0} \frac{(i \theta)^n}{n!} + \sum_{n \geq 0} (-1)^n \frac{(i \theta)^n}{n!}\right)$\\
$\tab[3.05cm] = \frac{1}{2} \left(\sum_{n \geq 0} \frac{(i \theta)^n}{n!} + \left(\sum_{n \mbox{ impair}} -\frac{(i \theta)^n}{n!} + \sum_{n \mbox{ pair}} \frac{(i \theta)^n}{n!}\right)\right)$\\
$\tab[3.05cm] = \frac{1}{2} \left(2 * \sum_{n \mbox{ pair}} \frac{(i \theta)^n}{n!}\right)$\\
$\tab[3.05cm] = \sum_{n \geq 0} \frac{(i \theta)^{2n}}{(2n)!}$\\
$\tab[3.05cm] = \sum_{n \geq 0} \frac{i^{2n}\theta^{2n}}{(2n)!}$\\
$\tab[3.05cm] = \sum_{n \geq 0} \frac{(-1)^{n}\theta^{2n}}{(2n)!}$\\
Même rayon de convergence que $e(x)$ $R = \infty$, et le domaine de convergence est $\mathbb{R}$\\

\newpage
\noindent6) $\sin(x)$\\
$\sin(\theta) = \frac{e^{i \theta} - e^{-i \theta}}{2i} = \frac{1}{2i} \left(\sum_{n \geq 0} \frac{(i \theta)^n}{n!} - \sum_{n \geq 0} \frac{(-i \theta)^n}{n!}\right)$\\
$\tab[3.05cm] = \frac{1}{2i} \left(\sum_{n \geq 0} \frac{(i \theta)^n}{n!} - \sum_{n \geq 0} (-1)^n \frac{(i \theta)^n}{n!}\right)$\\
$\tab[3.05cm] = \frac{1}{2i} \left(\sum_{n \geq 0} \frac{(i \theta)^n}{n!} - \left(\sum_{n \mbox{ impair}} -\frac{(i \theta)^n}{n!} + \sum_{n \mbox{ pair}} \frac{(i \theta)^n}{n!}\right)\right)$\\
$\tab[3.05cm] = \frac{1}{2i} \left(\sum_{n \geq 0} \frac{(i \theta)^n}{n!} - \left(-\sum_{n \mbox{ impair}} \frac{(i \theta)^n}{n!} + \sum_{n \mbox{ pair}} \frac{(i \theta)^n}{n!}\right)\right)$\\
$\tab[3.05cm] = \frac{1}{2i} \left(\sum_{n \geq 0} \frac{(i \theta)^n}{n!} + \sum_{n \mbox{ impair}} \frac{(i \theta)^n}{n!} - \sum_{n \mbox{ pair}} \frac{(i \theta)^n}{n!}\right)$\\
$\tab[3.05cm] = \frac{1}{2i} \left(2 * \sum_{n \mbox{ impair}} \frac{(i \theta)^n}{n!}\right)$\\
$\tab[3.05cm] = \frac{1}{i} \left(\sum_{n \geq 0} \frac{(i \theta)^{2n + 1}}{(2n + 1)!}\right)$\\
$\tab[3.05cm] = \frac{1}{i} \left(\sum_{n \geq 0} \frac{i^{2n + 1}\theta^{2n + 1}}{(2n + 1)!}\right)$\\
$\tab[3.05cm] = \frac{1}{i} \sum_{n \geq 0} i\frac{(-1)^{n}\theta^{2n + 1}}{(2n + 1)!}$\\
$\tab[3.05cm] = \sum_{n \geq 0} \frac{(-1)^{n}\theta^{2n + 1}}{(2n + 1)!}$\\
Même rayon de convergence que $e(x)$ $R = \infty$, et le domaine de convergence est $\mathbb{R}$\\
\\
7) $\cosh(x) =  \frac{e^{x} + e^{-x}}{2} = \frac{1}{2} \left(\sum_{n \geq 0} \frac{x^n}{n!} + \sum_{n \geq 0} \frac{(-x)^n}{n!}\right)$\\
$\tab[3.55cm] = \frac{1}{2} \left(2 * \sum_{\mbox{n pair}} \frac{x^n}{n!} \right)$\\
$\tab[3.55cm] = \sum_{n \geq 0} \frac{x^{2n}}{{(2n)}!}$\\
Même rayon de convergence que $e(x)$ $R = \infty$, et le domaine de convergence est $\mathbb{R}$

\newpage
\noindent 8) $\cosh(x) =  \frac{e^{x} - e^{-x}}{2} = \frac{1}{2} \left(\sum_{n \geq 0} \frac{x^n}{n!} - \sum_{n \geq 0} \frac{(-x)^n}{n!}\right)$\\
$\tab[3.55cm] = \frac{1}{2} \left(2 * \sum_{\mbox{n impair}} \frac{x^n}{n!} \right)$\\
$\tab[3.55cm] = \sum_{n \geq 0} \frac{x^{2n + 1}}{{(2n + 1)}!}$\\
Même rayon de convergence que $e(x)$ $R = \infty$, et le domaine de convergence est $\mathbb{R}$\\
\\
9) $arctan(x)'$ = $\frac{1}{1 + x^2}$\\
Soit $x \in ]-1, 1[$\\
On fait un changement de variable $X = -x^2$\\
$\frac{1}{1 - X} = \sum_{n \geq 0} X^n = \sum_{n \geq 0} (-x^2)^n = \sum_{n \geq 0} (-1)^n x^{2n}$\\
On a que: $\arctan(x) - \arctan(0) = \int_0^x \sum_{n \geq 0} (-1)^n t^{2n} dt = \sum_{n \geq 0} (-1)^n \frac{1}{2n + 1} x^{2n + 1}$\\
Donc $\arctan(x) = \sum_{n \geq 0} (-1)^n \frac{x^{2n + 1}}{2n + 1}$\\
Même rayon de convergence que $\frac{1}{1 - x}$, donc $R = 1$\\
Étude en $x = 1$\\
$\sum_{n \geq 0} (-1)^n \frac{1}{2n + 1}$ Converge critère des séries alternées\\
Étude en $x = -1$\\
$\sum_{n \geq 0} (-1)^n \frac{(-1)^{2n + 1}}{2n + 1} = \sum_{n \geq 0} (-1)^n \frac{-1}{2n + 1}$ Converge critère des séries alternées\\
Donc le domaine de convergence est $[-1, 1]$
\newpage
\noindent 10)
$\sin(\theta) = \frac{e^{i \theta} - e^{-i \theta}}{2i}$\\
$\sin(\theta)^2 = \left(\frac{e^{i \theta} - e^{-i \theta}}{2i}\right)^2$\\
$\tab[1.15cm] = -\frac{e^{2i\theta} -2e^{i\theta}e^{-i\theta} + e^{-2i\theta}}{4}$\\
$\tab[1.15cm] = -\frac{e^{2i\theta} + e^{-2i\theta} - 2}{4}$\\
$\tab[1.15cm] = -\frac{e^{2i\theta} + e^{-2i\theta}}{4} + \frac{1}{2}$\\
$\tab[1.15cm] = -\frac{1}{2}\left(\cos(2\theta) - 1\right)$\\
$\tab[1.15cm] = -\frac{1}{2}\left(\sum_{n \geq 0} \frac{(-1)^{n}(2\theta)^{2n}}{(2n)!} - 1\right)$\\
$\tab[1.15cm] = -\frac{1}{2}\left(\sum_{n \geq 1} \frac{(-1)^{n}(2\theta)^{2n}}{(2n)!}\right)$\\
\end{document}
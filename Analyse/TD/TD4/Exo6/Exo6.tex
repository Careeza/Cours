\documentclass{article}
\usepackage[utf8]{inputenc}
\usepackage{amsfonts}
\usepackage{amsmath}
\usepackage{graphicx}
\usepackage[a4paper, total={6in, 8in}]{geometry}
\usepackage{setspace}

\everymath{\displaystyle}

\newcommand\tab[1][1cm]{\hspace*{#1}}
\doublespacing
\author{Frederic Becerril}

\newcommand{\mylim}[2]{\underset{#1 \rightarrow #2}{\longrightarrow}}
\newcommand{\mysim}[2]{\underset{#1 \rightarrow #2}{\sim}}
\newcommand{\mysupp}[1]{\underset{x \in #1}{\sup}}
% \newcommand{\citer}[2]{\og #2 \fg{} (#1)}

\begin{document}

\part*{Exerice 6}

$S(x) = \sum_{n \geq 2} \frac{x^n}{\ln(n)}$\\
$a_n = \frac{1}{\ln(n)}$\\
Étudions $\frac{a_{n+1}}{a_n} = \frac{\ln(n + 1)}{\ln(n)} = \frac{\ln(n(1 + \frac{1}{n}))}{\ln(n)} = \frac{ln(n) + ln(1 + \frac{1}{n})}{ln(n)} = 1 + \frac{ln(1 + \frac{1}{n})}{ln(n)}$\\
Or $\frac{1}{n} \mylim{n}{\infty} 0$ donc ont peut faire un DL en 0 de $\ln(1 + u)$\vspace{2mm}\\
$\frac{ln(1 + \frac{1}{n})}{ln(n)} = \frac{\frac{1}{n} + o(\frac{1}{n})}{ln(n)} = \frac{1}{n\ln(n)} + o\left(\frac{1}{n\ln(n)}\right) \mylim{n}{\infty} 0$\vspace{2mm}\\
Donc $\frac{a_{n + 1}}{a_n} \mylim{n}{\infty} 1 = l$\\
Donc le critère de d'Alembert nous dis que $R = \frac{1}{l} = 1$\\
Étudions $S(x)$ en $\{-1, 1\}$\\
$S(-1)$ = $\sum_{n \geq 2} \frac{(-1)^n}{ln(n)}$ or grâce au critère spécial des séries alternées on a que:\\
$S(-1) = \sum_{n \geq 2} \frac{x^n}{ln(n)}$ Converge\\
$S(1) = \sum_{n \geq 2} \frac{1}{ln(n)}$\\ Or Bertrand nous dis que cette série diverge\\
Donc le domaine de convergence de cette série est $[-1, 1[$\\
$S(x) = \sum_{n \geq 2} \frac{x^n}{ln(n)}$\\
$ln(n) < n$ $\forall n \geq 2$, donc $\frac{1}{ln(n)} \geq \frac{1}{n}$\\
$x \in [0, 1[$, donc $\frac{x^n}{ln(n)} \geq \frac{x^n}{n}$ $\forall n \geq 2$\\
Donc $S(x) = \sum_{n \geq 2} \frac{x^n}{ln(n)} \geq \sum_{n \geq 2} \frac{x^n}{n}$\\
Or $\sum_{n \geq 2} \frac{x^n}{n} = -ln(1 - x)$, sur $[0, 1[$\\
Et $-ln(1 - x) \mylim{x}{1^-} \infty$\\
Donc $S(x) \mylim{x}{1^-} \infty$\\
\newpage
% \noindent3- $\frac{1}{\ln(n - 1)} = \frac{1}{\ln(n) + \ln(1 - \frac{1}{n})} = \frac{1}{\ln(n)\left(1 + \frac{ln(1 - \frac{1}{n})}{\ln(n)}\right)}$\\
% $\frac{1}{\ln(n - 1)} - \frac{1}{\ln(n)} = \frac{1}{\ln(n)\left(1 + \frac{ln(1 - \frac{1}{n})}{\ln(n)}\right)} - \frac{1}{ln(n)}$\vspace{3mm}\\
% $\tab[2.9cm] = \frac{1 - \left(1 + \frac{ln(1 - \frac{1}{n})}{\ln(n)}\right)}{\ln(n)\left(1 + \frac{ln(1 - \frac{1}{n})}{\ln(n)}\right)}$\vspace{3mm}\\
% $\tab[2.9cm] = \frac{-\frac{ln(1 - \frac{1}{n})}{\ln(n)}}{ln(n - 1)}$\vspace{3mm}\\
% $\tab[2.9cm] = \frac{-ln(1 - \frac{1}{n})}{ln(n)ln(n - 1)}$ On fait un DL en 0 de $ln(1 - x)$ \vspace{3mm} \\
% $\tab[2.9cm] = \frac{\frac{1}{n} + o(\frac{1}{n})}{ln(n)ln(n - 1)}$\vspace{3mm}\\
% $\tab[2.9cm] = \frac{1}{n\ln(n)\ln(n - 1)} + o\left(\frac{1}{n\ln(n)\ln(n - 1)}\right)$\vspace{3mm}\\
% Donc $\frac{1}{\ln(n - 1)} - \frac{1}{\ln(n)} \mysim{n}{\infty} \frac{1}{n\ln(n)\ln(n - 1)}$
\noindent3- a) $\sum_{n \geq 3} \frac{1}{\ln(n - 1)} - \frac{1}{\ln(n)}$\\
Passon par une somme partielle\\
$\sum_{k = 3}^n \frac{1}{\ln(k - 1)} - \frac{1}{\ln(k)} = \sum_{k = 3}^n \frac{1}{\ln(k - 1)} - \sum_{k = 3}^n \frac{1}{\ln(k)}$\\
$\tab[3.5cm] =  \sum_{k = 2}^{n - 1} \frac{1}{\ln(k)} -  \sum_{k = 3}^n \frac{1}{\ln(k)}$\\
$\tab[3.5cm] =  \frac{1}{\ln(2)} + \sum_{k = 3}^{n - 1} \frac{1}{\ln(k)} -  \left(\sum_{k = 3}^{n -1} \frac{1}{\ln(k)} + \frac{1}{ln(n)}\right)$\\
$\tab[3.5cm] =  \frac{1}{\ln(2)} - \frac{1}{\ln(n)}$\\
On a que $\sum_{n = 3}^\infty a_n = \underset{n \rightarrow \infty}{\lim} \sum_{k = 3}^n a_k$\\
$\sum_{n = 3}^\infty a_n = \underset{n \rightarrow \infty}{\lim} \left(\frac{1}{\ln(2)} - \frac{1}{\ln(n)}\right) = \frac{1}{ln(2)}$\\
b) On a que $a_n \geq 0$, donc $a_n x^n$ est strictement croissant sur $[0, 1]$\\
De plus on a que $|a_n (-x)^n|$ = $|a_n| |(-1)^n x^n| = |a_n| |(-1)^n| |x^n| = |a_n x^n|$\\
Donc si $\sum_{n \geq 3} a_n x^n$ converge normalement sur $[0, 1]$ alors $\sum_{n \geq 3} a_n x^n$ converge sur $[-1, 1]$\\
$\sum_{n \geq 3} a_n x_n$ converge normalement si $\sum_{n \geq 3} \mysupp{[0, 1]} |a_n x_n|$ conerge\\
$\sum_{n \geq 3} \mysupp{[0, 1]} |a_n x_n| = \sum_{n \geq 3} |a_n| = \sum_{n \geq 3} a_n$ qui converge\\
Donc $\sum_{n = 3}^\infty a_n x_n$ converge normalement sur $[-1, 1]$\\
\\
4- a) $x \in [-1, 1[$\\
$(x - 1)S(X) = (x - 1)\sum_{n \geq 2} \frac{x^n}{\ln(n)} = \sum_{n \geq 2} \frac{x^{n + 1}}{\ln(n)} - \sum_{n \geq 2} \frac{x^n}{\ln(n)}$\\
$\tab[2cm] =\sum_{n \geq 3} \frac{x^{n}}{\ln(n - 1)} - \sum_{n \geq 2} \frac{x^n}{\ln(n)}$\\
$\tab[2cm] =\sum_{n \geq 3} \frac{x^{n}}{\ln(n - 1)} - \left(\sum_{n \geq 3} \left(\frac{x^n}{\ln(n)}\right) + \frac{x^2}{ln(2)})\right)$\\
$\tab[2cm] =-\frac{x^2}{ln(2)} + \sum_{n \geq 3} \frac{x^{n}}{\ln(n - 1)} - \frac{x^{n}}{\ln(n)}$\\
$\tab[2cm] =-\frac{x^2}{ln(2)} + \sum_{n \geq 3} x^n \left(\frac{1}{\ln(n - 1)} - \frac{1}{\ln(n)}\right)$\\
$\tab[2cm] =-\frac{x^2}{ln(2)} + \sum_{n \geq 3} x^n a_n = -\frac{x^2}{ln(2)} + u(x)$\\
b) On a que $u(x)$ est continue sur $[-1 , 1]$ donc soit $x_n \in \mathbb{R}$ tq $x_n \mylim{n}{\infty} 1^-$\\
$u(x_n) \mylim{n}{\infty} u(1)$\\
Or $u(1) = \sum_{n = 3}^\infty a_n = \frac{1}{\ln(2)}$\\
Donc $\underset{x \rightarrow 1^-}{lim} (x - 1)S(x) = \underset{x \rightarrow 1^-}{lim} \left(-\frac{x^2}{ln(2)} + u(x)\right)$\\
$\tab[3.6cm] = -\frac{1}{\ln(2)} + \frac{1}{\ln(2)} = 0$ 
\end{document}
\documentclass{article}
\usepackage[utf8]{inputenc}
\usepackage{amsfonts}
\usepackage{amsmath}
\usepackage{graphicx}
\usepackage[a4paper, total={6in, 8in}]{geometry}
\usepackage{setspace}

\everymath{\displaystyle}

\newcommand\tab[1][1cm]{\hspace*{#1}}
\doublespacing
\author{Frederic Becerril}

\newcommand{\mylim}[2]{\underset{#1 \rightarrow #2}{\longrightarrow}}
\newcommand{\mysim}[2]{\underset{#1 \rightarrow #2}{\sim}}
% \newcommand{\citer}[2]{\og #2 \fg{} (#1)}

\begin{document}

\part*{Exerice 2}

$(a_n)_{n \in \mathbb{N}} \mylim{n}{\infty} = 0$\\
$\sum_{n \in \mathbb{N}} a_n$ diverge\\
Soit la série entière $\sum a_n x_n$\\
R = $sup \{x\in \mathbb{R}_+ \;|\; a_n x_n \mbox{ borné }\}$\\
Si $x = 1$ on $a_n x^n = a_n \mylim{n}{\infty} 0$ et donc $a_n x_n$ est borné\\
Si $x < 1$ donc $x^n \mylim{n}{\infty} 0$ et $a_n x^n \mylim{n}{\infty} 0$ donc borné\\
Si $x > 1$ on a $x^n \mylim{n}{\infty} \infty$ mais $a_n x_n \mylim{n}{\infty}$ est indeterminé\\
Mais on sait que $R \geq 1$\\
On sait aussi que R = $sup \{x\in \mathbb{R}_+ \;|\; \sum |a_n| x_n \mbox{ converge }\}$\\
Comme $\sum a_n$ diverge, $\sum |a_n|$ diverge aussi\\
or $\sum |a_n| \leq \sum |a_n| x_n$ quand $x > 1$\\
or comme $\sum |a_n|$ diverge $\sum |a_n| x_n$ diverge aussi quand $x > 1$\\
Donc on sait que $R \leq 1$\\
Finalement on obtient $R = 1$

\end{document}
\documentclass{article}
\usepackage[utf8]{inputenc}
\usepackage{amsfonts}
\usepackage{amsmath}
\usepackage{graphicx}
\usepackage[a4paper, total={6in, 8in}]{geometry}
\usepackage{setspace}
\usepackage{enumitem}

\everymath{\displaystyle}

\newcommand\tab[1][1cm]{\hspace*{#1}}
\doublespacing
\author{Frederic Becerril}

\newcommand{\mylim}[2]{\underset{#1 \rightarrow #2}{\longrightarrow}}
\newcommand{\mysim}[2]{\underset{#1 \rightarrow #2}{\sim}}
% \newcommand{\citer}[2]{\og #2 \fg{} (#1)}

\begin{document}

\part*{Exerice 1}

\begin{enumerate}[label=\alph*)]
    \item $f_n(x) = x^n$\\
        Soit $x \in \mathbb{R}$\\
        Si $x > 1 \Rightarrow x^n \mylim{n}{\infty} \infty$ donc $f_n$ ne converge pas simplement\\
        Si $x = 1 \Rightarrow x^n = 1 \mylim{n}{\infty} 1$ donc $f_n$ converge simplement\\
        Si $0 \leq x < 1 \Rightarrow x^n \mylim{n}{\infty} 0$ donc $f_n$ converge simplement\\
        Si $-1 < x < 0 \Rightarrow x^n \mylim{n}{\infty} 0$ donc $f_n$ converge simplement\\
        Si $x \leq -1 \Rightarrow x^n$ n'admet pas de limite et donc $f_n$ ne converge pas simplement\\
        Donc $f_n$ converge simplement si $x \in ]-1, 1]$
    \item $f_n(x) = \frac{x^n}{n}$\\
        Soit $x \in \mathbb{R}$\\
        Si $x > 1 \Rightarrow x^n \mylim{n}{\infty} \infty$ et par croissance comparé on a que $\frac{x^n}{n} \mylim{n}{\infty} \infty$\\
        Si $x = 1 \Rightarrow x^n = 1 \mylim{n}{\infty} 1$ donc $f_n \mylim{n}{\infty} 0$\\
        Si $0 \leq x < 1 \Rightarrow x^n \mylim{n}{\infty} 0$ donc $f_n \mylim{n}{\infty} 0$\\
        Si $-1 < x < 0 \Rightarrow x^n \mylim{n}{\infty} 0$ donc  $f_n \mylim{n}{\infty} 0$\\
        Si $x = -1 \Rightarrow \frac{x^n}{n} = \frac{(-1)^n}{n} \mylim{n}{\infty} 0$\\
        Si $x \leq 1 \Rightarrow x^n \mylim{n}{\infty} \pm \infty$ et par croissance comparé on a que $\frac{x^n}{n} \mylim{n}{\infty} \pm \infty$\\
        Donc $f_n$ converge simplement si $x \in [-1, 1]$
    \item  $f_n(x) = n^x$\\
        Soit $x \in \mathbb{R}$\\
        Si $x > 0 \Rightarrow n^x \mylim{n}{\infty} \infty$, donc $f_n$ ne converge pas simplement\\
        Si $x = 0 \Rightarrow n^0 = 1 \mylim{n}{\infty} 1$ donc $f_n$ converge simplement\\
        Si $x < 0$ soit $k = |x|$ on a $n^x = \frac{1}{n^k}$ or $k \geq 0$ donc $n^k \mylim{n}{\infty} \infty \Rightarrow f_n(x) = \frac{1}{n^k} \mylim{n}{\infty} 0$\\
        Donc $f_n$ converge simplement si $x \in ]-\infty, 0]$
    \newpage    
    \item $f_n(x) = x^n e^n = (xe)^n$\\
    Soit $x \in \mathbb{R}$\\
    On a vu a la question que $f_n(x) = x^n$ converge simplement si $x \in ]-1, 1]$\\
    Donc $-1 < ex \leq 1 \Leftrightarrow \frac{-1}{e} < x \leq \frac{1}{e}$\\
    Donc $f_n$ converge simplement si $x \in ]-\frac{1}{e}, \frac{1}{e}]$
    \item $f_n(x) = \frac{sin(n^2x)}{n}$\\
    Soit $x \in \mathbb{R}$\\
    On a que $\forall x \in \mathbb{R}$ $-1 \leq sin(n^2x) \leq 1$\\
    Donc $\forall x \in \mathbb{R}$ $\frac{sin(n^2x)}{n} \mylim{n}{\infty} 0$\\
    Donc $f_n$ converge simplement si $x \in \mathbb{R}$
    \item $f_n(x) = n sin(\frac{x}{n})$
    Soit $x \in \mathbb{R}$\\
    On a $\frac{x}{n} \mylim{n}{\infty} 0$ donc on peut faire un DL de sin en 0\\
    $sin(\frac{n}{x}) = \frac{x}{n} + o(\frac{1}{n})$\\
    $f_n(x) = n (\frac{x}{n} + o(\frac{1}{n}) = x + o(1) \mylim{n}{\infty} x$\\
    Donc $f_n$ converge simplement si $x \in \mathbb{R}$
    \item $f_n(x) = n^2(cos(\frac{x}{n}) - 1)$
    Soit $x \in \mathbb{R}$\\
    On a $\frac{x}{n} \mylim{n}{\infty} 0$ donc on peut faire un DL de cos en 0\\
    $cos(\frac{n}{x}) = 1 - \frac{x^2}{2n^2} + o(\frac{1}{n^2})$\\
    $f_n(x) = n^2 (\frac{x^2}{2n^2} + o(\frac{1}{n^2}) = \frac{x^2}{2} + o(1) \mylim{n}{\infty} \frac{x^2}{2}$\\
    Donc $f_n$ converge simplement si $x \in \mathbb{R}$
\end{enumerate}
\newpage
\begin{enumerate}[label=\alph*)]
    \item $f_n(x) \Rightarrow \left\{
        \begin{array}{ll}
            0 \mbox{ si x} \in ]-1, 1[\\
            1 \mbox{ si x} = 1\\
        \end{array}
    \right.$
    \item $f_n(x) \Rightarrow \left\{
        \begin{array}{ll}
            0 \mbox{ si x} \in [-1, 1]\\
        \end{array}
    \right.$
    \item $f_n(x) \Rightarrow \left\{
        \begin{array}{ll}
            0 \mbox{ si x} \in ]-\infty, 0[\\
            1 \mbox{ si x} = 0\\
        \end{array}
    \right.$
    \item $f_n(x) \Rightarrow \left\{
        \begin{array}{ll}
            0 \mbox{ si x} \in ]-\frac{1}{e}, \frac{1}{e}[\\
            1 \mbox{ si x} = \frac{1}{e}\\
        \end{array}
    \right.$
    \item $f_n(x) \Rightarrow \left\{
        \begin{array}{ll}
            0 \mbox{ si x} \in \mathbb{R}\\
        \end{array}
    \right.$
    \item $f_n(x) \Rightarrow \left\{
        \begin{array}{ll}
            x \mbox{ si x} \in \mathbb{R}\\
        \end{array}
    \right.$
    \item $f_n(x) \Rightarrow \left\{
        \begin{array}{ll}
            \frac{x^2}{2} \mbox{ si x} \in \mathbb{R}\\
        \end{array}
    \right.$
\end{enumerate}

\end{document}
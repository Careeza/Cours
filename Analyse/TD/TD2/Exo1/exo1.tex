\documentclass{article}
\usepackage[utf8]{inputenc}
\usepackage{amsfonts}
\usepackage{amsmath}
\usepackage{graphicx}
\usepackage[a4paper, total={6in, 8in}]{geometry}
\usepackage{setspace}
\usepackage{enumitem}
\usepackage{xparse}
\usepackage{tkz-tab}
\usepackage{amssymb}


\everymath{\displaystyle}

\newcommand\tab[1][1cm]{\hspace*{#1}}
\doublespacing
\author{Frederic Becerril}

\NewDocumentCommand{\mylim}{ O{n} O{\infty}}{\underset{#1 \rightarrow #2}{\longrightarrow}}
\NewDocumentCommand{\mynlim}{ O{n} O{\infty}}{\underset{#1 \rightarrow #2}{\nrightarrow}}
\newcommand{\mysim}[2]{\underset{#1 \rightarrow #2}{\sim}}
\newcommand{\mysupp}[1]{\underset{x \in #1}{Sup}}

\begin{document}

\part*{Exerice 1}

$u_n : [0, \infty[ \ni x \longmapsto \frac{arctan(nx)}{n^2}$, $n \geq 1$\\
Pour montrer que $\sum_{n \geq 1} u_n$ converge normalement il faut montrer que:\\
$\sum_{n \geq 1} \mysupp{\mathbb{R}_+} |u_n|$ converge, or on a que $0 < arctan(x)$ $\nearrow$ $\frac{\pi}{2}$\\
Donc $\mysupp{\mathbb{R}_+} |u_n| = \frac{\pi}{2n^2}$\\
Or $\sum_{n \geq 1} \frac{\pi}{n^2}$ converge d'après Riemann, donc $\sum_{n \geq 1} u_n$ converge normalement\\
$(u_n)' = n\frac{1}{1 + n^2x^2} * \frac{1}{n^2} = \frac{1}{n + n^3x^2}$\\
Soit $a > 0$
On a que $0 < (u_n)'$ décroissant sur $\mathbb{R}_+$\\
Donc $\mysupp{[a, \infty[} |(u_n)'| = (u_n)'(a) = \frac{1}{n + a^2n^3} \mysim{n}{\infty} \frac{1}{a^2n^3}$\\
$\sum_{n\geq 1} \mysupp{[a, \infty[} |(u_n)'|$ a le même comportement que $\sum_{n\geq 1} \frac{1}{a^2n^3}$\\
Or d'après Riemann cette série converge, donc $(u_n)'$ converge normalement sur $[0, \infty[$\\
Soit $S = \sum_{n \geq 1} u_n$\\
On a: $\left\{
    \begin{array}{ll}
        \mbox{S converge normalement sur } \mathbb{R}_+\\
        u_n \in C^0 \mbox{ sur } \mathbb{R}_+\\
    \end{array}
\right. \Rightarrow S$ est de la classe $C^0$ sur $\mathbb{R}_+$

On a: $\left\{
    \begin{array}{ll}
        \mbox{S converge simplement sur } [a, +\infty[\\
        \mbox{S' converge normalement sur } [a, +\infty[\\
        u_n \in C^1 \mbox{ sur } [a, +\infty[\\
    \end{array}
\right. \Rightarrow S$ est de la classe $C^1$ sur $[a, +\infty[$\\
Comme S est de classe $C^1$ sur tout intervalle $[a, +\infty[$ on a que S est de classe $C^1$ sur $]0, +\infty[$

\end{document}
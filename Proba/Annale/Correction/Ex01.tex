\documentclass{article}
\usepackage[utf8]{inputenc}
\usepackage{amsfonts}
\usepackage{amsmath}
\usepackage{graphicx}
\usepackage[a4paper, total={6in, 8in}]{geometry}
\usepackage{setspace}

\newcommand\tab[1][1cm]{\hspace*{#1}}
\doublespacing
\author{Frederic Becerril}

\begin{document}

\part*{Exercice 1}

1) A et B indépendant si $P(A \cap B) = P(A)P(B)$\\
2) A, B et C indépendant si
\begin{itemize}
    \item $P(A \cap B) = P(A)P(B)$
    \item $P(A \cap C) = P(A)P(C)$
    \item $P(B \cap C) = P(B)P(C)$
    \item $P(A \cap B \cap C) = P(A)P(B)P(C)$
\end{itemize}
3) Le nombre de parties a p éléments de n éléments est $\begin{pmatrix}
    n\\
    p\\
\end{pmatrix}$\\
4) On sait que $Card(P(E)) = 2^n$\\
donc $Card(P(E) \backslash {\emptyset}) = 2^n -1$\\
On peut aussi dire $Card(P(E) \backslash {\emptyset}) = \sum_{p=1}^n \begin{pmatrix}
    n\\
    p\\
\end{pmatrix} = \sum_{p=0}^n \begin{pmatrix}
    n\\
    p\\
\end{pmatrix} - \begin{pmatrix}
    n\\
    0\\
\end{pmatrix} = 2^n - 1$\\
5) $f : E \rightarrow F$\\
Si $\forall y \in F, card(f^{-1}(\{y\})) = n$\\
Alors $Card(E) = n card(F)$\\
6) $card(\{(i_j)_{1 \leq j \leq n} \in \{1, \dots, n\}^p, 1 \leq i_1 < i_2 < \dots < i_p \leq n\})
= \begin{pmatrix}
    n\\
    p\\
\end{pmatrix}$\\
7) $card(\{(i_j)_{1 \leq j \leq n} \in \{1, \dots, n\}^p, 1 \leq i_1 \leq i_2 < \dots < i_p \leq n\})$\\
$= card(\{(i_j)_{1 \leq j \leq n} \in \{1, \dots, n\}^p, 1 \leq i_1 < i_2 < \dots < i_p \leq n\})$ + \\
$\tab[1.3cm] card(\{(i_j)_{1 \leq j \leq n} \in \{1, \dots, n\}^p, 1 \leq i_1 = i_2 < \dots < i_p \leq n\})$\\
$= \begin{pmatrix}
    n\\
    p\\
\end{pmatrix} - \begin{pmatrix}
    n\\
    p - 1\\
\end{pmatrix}$\\
8) Le nombre de façon de choisir les 3 pile est $\begin{pmatrix}
    6\\
    3\\
\end{pmatrix}$ et le cardinal total est $2^6$\\
Donc la probabilité est $\frac{\begin{pmatrix}
    6\\
    3\\
\end{pmatrix}}{2^6} = \frac{20}{2^6}$
\end{document}
\documentclass{article}
\usepackage[utf8]{inputenc}
\usepackage{amsfonts}
\usepackage{amsmath}
\usepackage{graphicx}
\usepackage[a4paper, total={6in, 8in}]{geometry}
\usepackage{setspace}

\newcommand\tab[1][1cm]{\hspace*{#1}}
\doublespacing
\author{Frederic Becerril}

\begin{document}

\part*{Exercice 2}

1) Il y a $\begin{pmatrix}
    3\\
    14\\
\end{pmatrix}$ façon de choisir 3 chapitres\\
2) Il y a $\begin{pmatrix}
    k - 1\\
    2\\
\end{pmatrix}$ façon de choisir les 2 plus petit chapitres\\
3) $\begin{pmatrix}
    14\\
    3\\
\end{pmatrix} = Card(\{(i_j)_{1 \leq j \leq 3} 1 \leq i_1 < i_2 < i_3 \leq 14\})$\\
$\tab[1.5cm] = Card(\cup_{k = 3}^{14} (\{(i_j)_{1 \leq j \leq 3} 1 \leq i_1 < i_2 < i_3 = k\}))$\\
$\tab[1.5cm] = \sum_{k = 3}^{14} Card(\{(i_j)_{1 \leq j \leq 2}, 1 \leq i_1 \leq i_2 \leq k - 1\})$\\
$\tab[1.5cm] = \sum_{k = 3}^{14} \begin{pmatrix}
    k - 1\\
    2
\end{pmatrix}$\\
4) $\varphi : \left\{
    \begin{array}{ll}
        \{(i_j), 1 \leq i_1 < \dots < i_{k+1} \leq n + 1\} \rightarrow \cap_{p=k}^n \{(i_j), 1 \leq i_1 < \dots < i_k \leq p\} \times \{i_{k + 1} = p + 1\} (E \rightarrow F)\\
        (i_1, \dots, i_{k+1}) \longmapsto ((i_1, \dots, i_k), i_{k+1})
    \end{array}
\right.$
$\varphi$ est bien définie ?\\
Oui car $i_{k+1} \in \{k + 1, \dots, n+1\}$ et si $i_{k + 1} = p + 1, (i_1, \dots, i_k) \in \{(i_j), 1 \leq i_1 < \dots < i_k \leq p\}$
$\varphi$ est injective ? Oui.\\
$\varphi$ est surjective ?\\
Soit $((i_1, \dots, i_k), i_{k+1}) \in F$\\
alors $1 \leq i_1 < i_2 < \dots < i_k \leq i_{k + 1} - 1 < i_{k + 1} \leq n$\\
Donc $(i_1, \dots, i_{k+1}) \in E$\\
Et $\varphi((i_1, \dots, i_{k+1})) = ((i_1, \dots, i_k), i_{k+1})$\\
Donc $\varphi$ est surjective\\
Donc $\varphi$ est bijective\\
Finalement on a $\begin{pmatrix}
    n + 1\\
    k + 1\\
\end{pmatrix} = Card(E)$\\
$\tab[4.1cm] = Card(F)$\\
$\tab[4.1cm] = \sum_{p = k}^n \begin{pmatrix}
    p\\
    k\\
\end{pmatrix}$


\end{document}
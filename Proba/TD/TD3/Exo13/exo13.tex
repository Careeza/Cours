\documentclass{article}
\usepackage[utf8]{inputenc}
\usepackage{amsfonts}
\usepackage{amsmath}
\usepackage{graphicx}
\usepackage[a4paper, total={6in, 8in}]{geometry}
\usepackage{setspace}

\everymath{\displaystyle}

\newcommand\tab[1][1cm]{\hspace*{#1}}
\doublespacing
\author{Frederic Becerril}

\newcommand{\mylim}[2]{\underset{#1 \rightarrow #2}{\longrightarrow}}
\newcommand{\mysim}[2]{\underset{#1 \rightarrow #2}{\sim}}
% \newcommand{\citer}[2]{\og #2 \fg{} (#1)}

\begin{document}

\part*{Exerice 13}

Pierre a une chaussette mais a égaré la deuxième. Celle-ci a une probabilité p $(p > 0.1)$ d'être
dans sa commode, qui comporte n tiroirs, et peut être dans n'importe quel tiroir. Pierre décide
d'ouvrir k tiroirs $(1 \leq k \leq n)$ : s'il trouve sa chaussette, c'est qu'elle était dans la commode ;
s'il ne la trouve pas, il se dit que sans doute elle n'y était pas. On aimerait connaître le nombre
minimum de tiroirs à ouvrir pour que Pierre se trompe avec une probabilité inférieure à 0.1.\\
T l'événement "Pierre se trompe."\\
C l'événement "La chaussette est dans la commode."
\begin{enumerate}
    \item $P(T|C) = \frac{n - k}{n}$ \tab $P(T|C^c) = 0$\\
        On a que $P(T) = P(T|C)P(C) + P(T|C^c)P(C^c) = P(T|C)P(C) = \frac{n - k}{n}p$
    \item On veut que $P(T) < 0.1 \Leftrightarrow \frac{n - k}{n}p < 0.1 \Leftrightarrow (n - k)p < 0.1n$\\
        $\tab[3.85cm] \Leftrightarrow n - k < \frac{0.1n}{p} \Leftrightarrow k > n - \frac{0.1n}{p}$\\
        Donc on prend k = $\lceil n(1 - \frac{0.1}{p}) \rceil$
    \item On a pris $p > 0.1$ car si $p < 0.1$ on a $\frac{0.1n}{p} > n$ donc $n - \frac{0.1n}{p} < 0$\\
    Donc k = 0 suffit
\end{enumerate}

\end{document}
\documentclass{article}
\usepackage[utf8]{inputenc}
\usepackage{amsfonts}
\usepackage{amsmath}
\usepackage{graphicx}
\usepackage[a4paper, total={6in, 8in}]{geometry}
\usepackage{setspace}

\everymath{\displaystyle}

\newcommand\tab[1][1cm]{\hspace*{#1}}
\doublespacing
\author{Frederic Becerril}

\newcommand{\mylim}[2]{\underset{#1 \rightarrow #2}{\longrightarrow}}
\newcommand{\mysim}[2]{\underset{#1 \rightarrow #2}{\sim}}
% \newcommand{\citer}[2]{\og #2 \fg{} (#1)}

\begin{document}

\part*{Exerice 12}

Un laboratoire a mis au point un alcootest dont les propriétés sont les suivantes :
\begin{itemize}
    \item il se révèle positif pour quelqu'un qui n'est pas en état d'ébriété dans 2\% des cas
    \item il se révèle positif pour quelqu'un qui est en état d'ébriété dans 96\% des cas.
\end{itemize}
Dans un département donné, on estime que 3\% des conducteurs sont en état d'ébriété.\\
Soit A = conducteurs en état d'ébriété\\
$A^c$ = conducteurs n'est pas en état d'ébriété\\
P = alcootest positif\\
N = alcootest négatif
\begin{enumerate}
    \item On cherche $P(A^c|P) = \frac{P(A^c \cap P)}{P(P)}$\\
    On sait que $P(A^c \cap P) = P(P|A^c)P(A^c) = \frac{2}{100} * \frac{97}{100} = \frac{194}{10000}$\\
    On a $P(P) = P(P|A)P(A) + P(P|A^c)P(A^c)$\\
    $\tab[1.8cm] = \frac{96}{100} * \frac{3}{100} + \frac{2}{100} * \frac{97}{100} = \frac{482}{10000}$\\
    Donc $P(A^c|P) = \frac{194}{10000} * \frac{10000}{482} = \frac{194}{482}$
    \item On cherche $P(A^c|N) =  \frac{P(N \cap A^c)}{P(N)}$\\
    On a que $P(N \cap A^c) = P(N|A^c) P(A^c) = \frac{98}{100} * \frac{97}{100} = \frac{9506}{10000}$\\
    Et $P(N) = 1 - P(P) = 1 - \frac{482}{10000} = \frac{9518}{10000}$\\ 
    Donc $P(N | A) = \frac{9506}{10000} * \frac{10000}{9518} = \frac{9506}{9518} \approx 99.8\%$

\end{enumerate}

\end{document}
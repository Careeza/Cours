\documentclass{article}
\usepackage[utf8]{inputenc}
\usepackage{amsfonts}
\usepackage{amsmath}
\usepackage{graphicx}
\usepackage[a4paper, total={6in, 8in}]{geometry}
\usepackage{setspace}

\everymath{\displaystyle}

\newcommand\tab[1][1cm]{\hspace*{#1}}
\doublespacing
\author{Frederic Becerril}

\newcommand{\mylim}[2]{\underset{#1 \rightarrow #2}{\longrightarrow}}
\newcommand{\mysim}[2]{\underset{#1 \rightarrow #2}{\sim}}
% \newcommand{\citer}[2]{\og #2 \fg{} (#1)}

\begin{document}

\part*{Exerice 11}

On dispose d'un lot de 100 pièces de monnaie toutes de même apparence. On sait que 50 de
ces pièces sont équilibrées tandis que les 50 autres sont truquées. Pour une pièce truquée, la
probabilité d'apparition de "face" lors d'un jet de cette pièce vaut $\frac{3}{4}$.\\
On cherche la probabilité d'obtenir face après un lancé d'une pièce au hasard\\
Soit P(F) la probabilité d'avoir face\\
Soit P(T) la probabilité que la pièce soit truquée\\
Soit P(NT) la probabilité que la pièce ne soit pas truquée\\
$P(F) = P(F \cap T) + P(F \cap NT)$\\
$\tab[0.9cm] = P(T) * P(F|T) + P(NT) * P(F|NT)$\\
$\tab[0.9cm] = \frac{1}{2} * \frac{3}{4} + \frac{1}{2} * \frac{1}{2} = \frac{5}{8}$\\
On cherche $P(T|F)$\\
$P(T|F) = \frac{P(T \cap F)}{P(F)} = \frac{\frac{1}{2} * \frac{3}{4}}{\frac{5}{8}} = \frac{3}{5}$


\end{document}
\documentclass{article}
\usepackage[utf8]{inputenc}
\usepackage{amsfonts}
\usepackage{amsmath}
\usepackage{graphicx}
\usepackage[a4paper, total={6in, 8in}]{geometry}
\usepackage{setspace}

\everymath{\displaystyle}

\newcommand\tab[1][1cm]{\hspace*{#1}}
\doublespacing
\author{Frederic Becerril}

\newcommand{\mylim}[2]{\underset{#1 \rightarrow #2}{\longrightarrow}}
\newcommand{\mysim}[2]{\underset{#1 \rightarrow #2}{\sim}}
% \newcommand{\citer}[2]{\og #2 \fg{} (#1)}

\begin{document}

\part*{Exerice 7}
CP : 4 garçon, 6 filles\\
CE1 : 6 garçon, n filles\\
Les deux classes dans une même salle\\
On choisit un élève au hazard:\\
Trouver n pour que les événements:
\begin{itemize}
    \item G = "l'élève est un garçon"
    \item CP = "l'élève est en CP"
\end{itemize}
Soit indépendant\\
On veut que $P(G \cap CP) = P(G)P(CP)$\\
$P(G) = \frac{10}{16 + n}$\\
$P(CP) = \frac{10}{16 + n}$\\
$P(G) P(CP) = \frac{100}{(16 + n)^2}$\\
$P(CP \cap G) = \frac{4}{16 + n}$\\
Donc $\frac{4}{16 + n} = \frac{100}{(16 + n)^2} \Leftrightarrow 4(16 + n) = 100 \Leftrightarrow n = 9$
\end{document}
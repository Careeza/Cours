\documentclass{article}
\usepackage[utf8]{inputenc}
\usepackage{amsfonts}
\usepackage{amsmath}
\usepackage{graphicx}
\usepackage[a4paper, total={6in, 8in}]{geometry}
\usepackage{setspace}

\newcommand\tab[1][1cm]{\hspace*{#1}}
\doublespacing
\author{Frederic Becerril}
\everymath{\displaystyle}

\begin{document}

\part*{Exercice 1}

Soit E, F et G trois sous-ensembles d'un ensemble $\Omega$. Représenter sur un dessin puis démontrer :
\begin{enumerate}
    \item $E \cap F \subset E \subset E \cup F$\\
soit $x \in E \cap F$ alors $x \in E$ et $x \in F$\\
Donc on a bien $E \cap F \subset E$\\
Soit $x \in E$ alors $x \in E \cup F$\\
Donc on a bien $E \subset E \cup F$
    \item $E \subset F$ alors $F^c \subset E^c$\\
Si $x \notin F$ alors $x \notin E$ car autrement E ne serait pas une partie F\\
Donc on a bien $F^c \subset E^c$
    \item $F \backslash E = F \cap E^c$\\
Si $x \in F \backslash E$, alors $x \in F$ et $x \notin E$\\
Donc on a bien $F \backslash E \subset F \cap E^c$\\
Si $x \in F \cap E^c$, alors $x \in F$ et $x \notin E$\\
Donc on a bien $F \cap E^c \subset F \backslash E$\\
Au final on a bien $F \backslash E = F \cap E^c$
    \item $F = (F \cap E) \cup (F \cap E^c)$\\
Soit $x \in F$, on a que soit $x \in E$ soit $x \in E^c$\\
Donc si $x \in E$, $x \in (F \cap E)$ et $x \in (F \cap E) \cup (F \cap E^c)$\\
Si $x \in E^c$, $x \in (F \cap E^c)$ et $x \in (F \cap E) \cup (F \cap E^c)$\\
Donc $F \subset (F \cap E) \cup (F \cap E^c)$\\
Soit $x \in (F \cap E) \cup (F \cap E^c)$\\
$x \in (F \cap E)$ ou $x \in (F \cap E^c)$\\
Si $x \in (F \cap E)$ alors $x \in F$ et $x \in E$ donc $x \in F$\\
Si $x \in (F \cap E^c)$ alors $x \in F$ et $x \in E^c$ donc $x \in F$\\
Donc $(F \cap E) \cup (F \cap E^c) \subset F$\\
Donc on a bien $F = (F \cap E) \cup (F \cap E^c)$
    \item TODO
    \item TODO
\end{enumerate}

\end{document}
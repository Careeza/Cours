\documentclass{article}
\usepackage[utf8]{inputenc}
\usepackage{amsfonts}
\usepackage{amsmath}
\usepackage{graphicx}
\usepackage[a4paper, total={6in, 8in}]{geometry}
\usepackage{setspace}

\newcommand\tab[1][1cm]{\hspace*{#1}}
\doublespacing
\author{Frederic Becerril}
\everymath{\displaystyle}

\begin{document}

\part*{Exercice 2}

Soit A et B deux sous-ensemble de $\Omega$\\
Soit $x \in \Omega$\\
Si on a $1_B(x) = 0$ alors $1_A(x) \leq 0 \Rightarrow 1_A(x) = 0$\\
Donc si $x \notin B$ alors $x \notin A$\\
$\Rightarrow B^c \subset A^c$ or on a vu a l'exercice 1.2 que $B^c \subset A^c$ alors $A^{c^c} \subset B^{c^c} \Rightarrow A \subset B$ 

\end{document}
\documentclass{article}
\usepackage[utf8]{inputenc}
\usepackage{amsfonts}
\usepackage{amsmath}
\usepackage{graphicx}
\usepackage[a4paper, total={6in, 8in}]{geometry}
\usepackage{setspace}
\usepackage{xcolor}

\everymath{\displaystyle}

\newcommand\tab[1][1cm]{\hspace*{#1}}
\doublespacing
\author{Frederic Becerril}

\newcommand{\important}[1]{{\color{red}\underline{\textbf{#1}}}}
\newcommand{\hbrace}[2]{\underset{#1}{\underbrace{#2}}}
\newcommand{\mylim}[2]{\underset{#1 \rightarrow #2}{\longrightarrow}}
\newcommand{\mysim}[2]{\underset{#1 \rightarrow #2}{\sim}}
% \newcommand{\citer}[2]{\og #2 \fg{} (#1)}
% \color{red}

\begin{document}

\part*{Exerice 2}

N lancers successifs d'une pièce déséquilibré. $\Omega = \{0, 1\}^N$\\
$(\omega_1, \dots, \omega_N) \in \{0, 1\}^N$\\
$P((\hbrace{\mbox{que des 0}}{0, \dots, 0})) = (1-p)^N$\\
$P((\hbrace{\mbox{que des 1}}{1, \dots, 1})) = p^N$\\
$P((1, \hbrace{\mbox{que des 0}}{0, \dots, 0})) = p (1-p)^{N -1}$\\
$P((\omega_1, \dots, \omega_N)) = p^{\fbox{$k$}}(1-p)^{\fbox{$N-k$}\mbox{||nombre de 0}}$, si k = $\sum_{i=1}^N \omega_i$\\
On regarde la somme:\\
$\tab[2cm] X:\Omega \rightarrow \{0, \dots, N\}$\\
$\tab \omega = (\omega_1, \dots, \omega_N) \mapsto \omega_1 + \dots + \omega_N$\\
$P(X = k) = P(\{\omega : \omega_1 + \dots + \omega_N = k\})$\\
$\tab[1.65cm] = \sum_{\omega_1 + \dots + \omega_N = k} P(w) = \sum_{\omega_1 + \dots + \omega_N = k} p^k(1-p)^{N-k}$ k = le nombre de succès \vspace{3mm}\\
$\tab[4.7cm] = \begin{pmatrix}
    N\\
    k\\
\end{pmatrix}  p^k(1-p)^{N-k}$\\
C'est la loi binomial BIN(N, P)\\
N = nombre de lancers\\
P = proba de succès\\
\important{NB:} si $p = \frac{1}{2}$, P uniforme sur $\{0, 1\}^N$ et alors $P(X = k) = \begin{pmatrix}
    N\\
    k\\
\end{pmatrix}2^{-N}$

\end{document}
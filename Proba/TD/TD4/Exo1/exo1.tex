\documentclass{article}
\usepackage[utf8]{inputenc}
\usepackage{amsfonts}
\usepackage{amsmath}
\usepackage{graphicx}
\usepackage[a4paper, total={6in, 8in}]{geometry}
\usepackage{setspace}

\everymath{\displaystyle}

\newcommand\tab[1][1cm]{\hspace*{#1}}
\doublespacing
\author{Frederic Becerril}

\newcommand{\mylim}[2]{\underset{#1 \rightarrow #2}{\longrightarrow}}
\newcommand{\mysim}[2]{\underset{#1 \rightarrow #2}{\sim}}
% \newcommand{\citer}[2]{\og #2 \fg{} (#1)}

\begin{document}

\part*{Exerice 1}

\section*{Correction 1}

On demande la loi de $1_A \Omega \rightarrow \{0, 1\}$\\
P(X = 1) = $P(\{\omega \in \Omega, 1_A(w) = 1\}) = P(A)$\\
et P(X = 0) = $P(\{\omega \in \Omega, 1_A(w) = 0\}) = P(A^c) = 1 - P(A)$\\
X a une loi de Bernouilli de paramètre P(A)

\section*{Correction 2}

($\Omega, P$) espace de proba finie.\\
$1_A$ : $\Omega \rightarrow \fbox{\{0, 1\}}$||s'appelle v.a de Bernouilli\\
$\tab[0.8cm] w \mapsto 1_A(w) = \left\{
    \begin{array}{ll}
        0 \mbox{ si } w \notin A\\
        1 \mbox{ si } w \in A\\
    \end{array}
\right.$\\
Soi: fonction de masse : $P(1_A = s)$, $\forall s \in \{0, 1\}$\\
$P(1_A = 1) = P(\{w: 1_A(w) = 1\}) = P(A)$\\
$P(1_A = 0) = P(\{w: 1_A(w) = 0\})$\\
$\tab[1.7cm] = P(\{w:w\notin A\}) = P(A^c) = 1 - P(A)$\\
On a donc que X = $1_A$ suit une loi de Bernouilli de paramètre $P(A)$


\end{document}
\documentclass{article}
\usepackage[utf8]{inputenc}
\usepackage{amsfonts}
\usepackage{amsmath}
\usepackage{graphicx}
\usepackage[a4paper, total={6in, 8in}]{geometry}
\usepackage{setspace}
\usepackage{xcolor}

\everymath{\displaystyle}

\newcommand\tab[1][1cm]{\hspace*{#1}}
\doublespacing
\author{Frederic Becerril}

\newcommand{\important}[1]{{\color{red}\underline{\textbf{#1}}}}
\newcommand{\hbrace}[2]{\underset{#1}{\underbrace{#2}}}
\newcommand{\mylim}[2]{\underset{#1 \rightarrow #2}{\longrightarrow}}
\newcommand{\mysim}[2]{\underset{#1 \rightarrow #2}{\sim}}
% \newcommand{\citer}[2]{\og #2 \fg{} (#1)}
% \color{red}

\begin{document}

\part*{Exerice 3}

$\Omega = \{0, 1\}^{\mathbb{N}}$\\
$\omega = (\omega_1, \omega_2, \dots) \in \{0, 1\}^{\mathbb{N}}$\\
On se munit de la variable aléatoire\\
$X: \Omega \rightarrow \mathbb{N} \cup \infty$\\
$\omega \mapsto $
$\left\{
    \begin{array}{ll}
        k \mbox{ si } \{\hbrace{\mbox{ k zeros }}{0, ..., 0,} \; 1\}\\
        \infty \mbox{ si }\{\hbrace{\infty}{0, ..., 0}\}\\
    \end{array}
\right.$\\
$P(X = k) = (1 - p)^{k - 1}p$\\



\end{document}
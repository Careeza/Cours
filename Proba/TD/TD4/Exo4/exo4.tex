\documentclass{article}
\usepackage[utf8]{inputenc}
\usepackage{amsfonts}
\usepackage{amsmath}
\usepackage{graphicx}
\usepackage[a4paper, total={6in, 8in}]{geometry}
\usepackage{setspace}
\usepackage{xcolor}

\everymath{\displaystyle}

\newcommand\tab[1][1cm]{\hspace*{#1}}
\doublespacing
\author{Frederic Becerril}

\newcommand{\important}[1]{{\color{red}\underline{\textbf{#1}}}}
\newcommand{\hbrace}[2]{\underset{#1}{\underbrace{#2}}}
\newcommand{\mylim}[2]{\underset{#1 \rightarrow #2}{\longrightarrow}}
\newcommand{\mysim}[2]{\underset{#1 \rightarrow #2}{\sim}}
% \newcommand{\citer}[2]{\og #2 \fg{} (#1)}
% \color{red}

\begin{document}

\part*{Exerice 4}

$\Omega = \{(x_1, \dots, x_p) \in \{1, \dots, N\} \mbox{ tq } \forall i \neq j, \mbox{ alors } x_i \neq x_j\}$\\
$\omega = (\omega_1, \dots, \omega_p) \in \Omega$\\
$X : \Omega \rightarrow \{0, \dots, n\}$\\
$\omega \mapsto \sum_{k = 1}^p  \left\{
    \begin{array}{ll}
        1 \mbox{ si } \omega_k \leq n\}\\
        0 \mbox{ sinon }\\
    \end{array}
\right.$\\
\\
$\forall k \in \{0, \dots, \min(n, p)\}$\\
$\mathbb{P}(X = k) = \frac{\begin{pmatrix}
    n\\
    k\\
\end{pmatrix} \begin{pmatrix}
    N - n\\
    p - k\\
\end{pmatrix}}{\begin{pmatrix}
    N\\
    p\\
\end{pmatrix}}$\\
Loi hypergéometrique de paramtètre $p, N, n$

\end{document}
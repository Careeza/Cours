\documentclass{article}
\usepackage[utf8]{inputenc}
\usepackage{amsfonts}
\usepackage{amsmath}
\usepackage{graphicx}
\usepackage[a4paper, total={6in, 8in}]{geometry}
\usepackage{setspace}

\newcommand\tab[1][1cm]{\hspace*{#1}}
\doublespacing
\author{Frederic Becerril}
\everymath{\displaystyle}

\begin{document}

\part*{DM 2}

$\bullet$ $\Omega = \{1, 2, 3, 4, 5, 6\}^n$ est l'ensemble des résultats possibles\\
Comme le dé est équilibré, la mesure de probabilité P est une proba uniforme on a:\\
Soit A une partie de $\Omega$ $P(A) = \frac{Card(A)}{Card(\Omega)}$\\
$\bullet$ L'experience "obenir un 6 a un lancé" suit la loi de Bernoulli\\
Donc l'experience "obtenir k 6 sur n lancés" suit la loi binomiale on a donc que\\
$P(A_k) = \begin{pmatrix}
    n\\
    k\\
\end{pmatrix} (\frac{1}{6})^k * (\frac{5}{6})^{n - k}$\\
Donc $P(A_0) = \begin{pmatrix}
    n\\
    0\\
\end{pmatrix} (\frac{1}{6})^0 * (\frac{5}{6})^{n} = (\frac{5}{6})^{n}$\\
Et $P(A_1) = \begin{pmatrix}
    n\\
    1\\
\end{pmatrix} (\frac{1}{6})^1 + (\frac{5}{6})^{n - 1} = \frac{n}{6} * (\frac{5}{6})^{n - 1}$\\
$\bullet$ $B_1$ est le complémentaire de $A_0$, car si obtient pas exactement 0 fois le chiffre 6
alors on a obtenu au moins un 6\\
$P(B_1) = 1 - P(A_0) = 1 - (\frac{5}{6})^n$\\
$\bullet$ Soit $A_{1p}$ l'événement on a obtenu au plus 1 fois le chiffre 6\\
On a $A_{1p} = A_0 \cup A_{1}$\\
$B_2$ est le complémentaire de $A_{1p}$, car si obtient pas au plus 1 fois le chiffre 6 alors on obtient au moins 2 fois le chiffre 6\\
$P(B_2) = P(A_{1p}^c) = 1 - P(A_{1p}) = 1 - P(A_0 \cup A_1)$\\
Or $A_0$ et $A_1$ sont des événement disjoints\\
$P(B_2) = 1 - (P(A_0) + P(A_1))$\\
$\tab = 1 - P(A_0) - P(A_1) = 1 - (\frac{5}{6})^n - \frac{n}{6} * (\frac{5}{6})^{n - 1}$\\
$\bullet$ On a déjà calculé la formule générale pour $A_k$\\
$P(A_k) = \begin{pmatrix}
    n\\
    k\\
\end{pmatrix} (\frac{1}{6})^k * (\frac{5}{6})^{n - k}$\\
\newpage
\noindent $\bullet$ Soit $A_{kp}$ l'événement on a obtenu au plus k fois le chiffre 6\\
$A_{kp} = \cup_{i=0}^{k} A_i$\\
$P(B_k) = P((A_{(k - 1)p})^c) = 1 - P(A_{(k - 1)p})  = 1 - P(\cup_{i=0}^{k-1} (A_i))$\\
Or tous les événements $A_k$ sont disjoints\\
Donc $P(B_k) = 1 - \sum_{i=0}^{k-1} P(A_i)$\\
$\tab[2cm] = 1 - \sum_{i=0}^{k-1} \begin{pmatrix}
    n\\
    i\\
\end{pmatrix} (\frac{1}{6})^i (\frac{5}{6})^{n - i}$
\end{document}
\documentclass{article}
\usepackage[utf8]{inputenc}
\usepackage{amsfonts}
\usepackage{amsmath}
\usepackage{graphicx}
\usepackage[a4paper, total={6in, 8in}]{geometry}
\usepackage{setspace}

\newcommand\tab[1][1cm]{\hspace*{#1}}
\doublespacing
\author{Frederic Becerril}
\everymath{\displaystyle}

\begin{document}

\part*{DM 1}

\begin{enumerate}
    \item Pour chaque tirage de boule on a 5 possibilitées et on tire 3 fois sans remise. Donc le nombre tirage possible est de $Card(\{1, 2, 3, 4 ,5\}^3) = Card(\{1, 2, 3, 4, 5\})^3 = 5^3 = 125$
    \item $(\{3\} \times \{1, 2, 3, 4 ,5\}^2) \cup (\{1, 2, 3, 4 ,5\} \times \{3\} \times \{1, 2, 3, 4 ,5\}) \cup (\{3\} \times \{1, 2, 3, 4 ,5\}^2)$
    \item On tire les trois boules simultanément donc on fait une partition de cardinal 3 de notre ensemble de boule de cardinal 5 donc il y a $\begin{pmatrix}
        5\\
        3\\
    \end{pmatrix}$ tirage possible, $\begin{pmatrix}
        5\\
        3\\
    \end{pmatrix} = 10$
    \item Si on considère que le tirage contient un 3, il nous faut regarder combien de partie il reste. Il nous faut tirer 2 boules simultanément dans un sac de 4 boules.\\
$Card(\{A \in P(E) | Card(A) = 3 \mbox{ et } 3 \in A\})$\\
=$Card(\{A \in P(E \backslash \{3\}) | Card(A) = 2\})$\\
=$\begin{pmatrix}
    4\\
    2\\
\end{pmatrix} = 6$
\end{enumerate}

\end{document}
\documentclass{article}
\usepackage[utf8]{inputenc}
\usepackage{amsfonts}
\usepackage{amsmath}
\usepackage{graphicx}
\usepackage[a4paper, total={6in, 8in}]{geometry}
\usepackage{setspace}

\everymath{\displaystyle}

\newcommand\tab[1][1cm]{\hspace*{#1}}
\doublespacing
\author{Frederic Becerril}

\NewDocumentCommand{\mylim}{ O{n} O{\infty}}{\underset{#1 \rightarrow #2}{\longrightarrow}}
\newcommand{\mysupp}[1]{\underset{x \in #1}{Sup}}
\newcommand{\mysim}[2]{\underset{#1 \rightarrow #2}{\sim}}
% \newcommand{\citer}[2]{\og #2 \fg{} (#1)}

\begin{document}

\part*{Exerice}

$\sum_{n \geq 1} (1 + \frac{1}{n})^{n^2} x^n$\\
$a_n = (1 + \frac{1}{n})^{n^2}$\\
Étudion $\sqrt[n]{a_n} = \sqrt[n]{(1 + \frac{1}{n})^{n^2}} = (1 + \frac{1}{n})^n$\\
Or $(1 + \frac{1}{n})^n \mylim e$\\
Donc d'après Cauchy on a $R = \frac{1}{e} = e^{-1}$\\
Si $x = e^{-1}$ on a:\\
$\sum_{n \geq 1} (1 + \frac{1}{n})^{n^2} e^{-n} = \sum_{n \geq 1} e^{n^2 ln(1 + \frac{1}{n})}e^{-n}$\\
$\tab[2.7cm] = \sum_{n \geq 1} e^{n^2 (\frac{1}{n} + \frac{1}{2n^2}o(\frac{1}{n^2})) -n}$\\
$\tab[2.7cm] = \sum_{n \geq 1} e^{n + \frac{1}{2} + o(1) -n}$\\
$\tab[2.7cm] = \sum_{n \geq 1} e^{\frac{1}{2} + o(1)}$ Cette série diverge

\end{document}